\PassOptionsToPackage{table,xcdraw}{xcolor}
\documentclass[10pt,conference]{IEEEtran}
\IEEEoverridecommandlockouts
% The preceding line is only needed to identify funding in the first footnote. If that is unneeded, please comment it out.

\usepackage{amsmath,amssymb,amsfonts}
%\usepackage{algorithmic}
\usepackage{graphicx}
\usepackage{textcomp}
\usepackage{xcolor}


\usepackage{cite}
\usepackage{booktabs}   %% For formal tables:
                        %% http://ctan.org/pkg/booktabs
\usepackage{subcaption} %% For complex figures with subfigures/subcaptions
                        %% http://ctan.org/pkg/subcaption
\usepackage{array}
%\usepackage{amsmath,amsfonts}
\usepackage{algorithm}
\usepackage[noend]{algpseudocode}
%\usepackage{algorithmic}
%\usepackage{graphicx}
%\usepackage{textcomp}
\usepackage{float}
\usepackage{listings}
\usepackage{xspace}
\usepackage{multirow}
\usepackage{amsthm}
\newtheorem{definition}{Definition}
\usepackage{balance}

\usepackage[skins]{tcolorbox}

\usepackage{xcolor,pifont}
\newcommand*\colourcheck[1]{%
	\expandafter\newcommand\csname #1check\endcsname{\textcolor{#1}{\ding{52}}}%
}
\colourcheck{blue}
\colourcheck{green}
\colourcheck{red}

\newtcolorbox{myframe}[2][]{%
  enhanced,colback=white,colframe=black,coltitle=black,
  sharp corners,
  toprule=1.0pt,
  rightrule=0.3pt,
  leftrule=0pt,
  bottomrule=0pt,
  fonttitle=\itshape\scshape\large,
  left=0pt,right=5pt,top=5pt,bottom=3pt,
  attach boxed title to top right={yshift=-0.3\baselineskip-0.4pt,xshift=-5mm},
  boxed title style={tile,size=minimal,left=0.2mm,right=0.5mm,
    colback=white,before upper=\strut},
  title=#2,#1
}

%\newcommand{\code}[1]{{\footnotesize\textsf{#1}}}

\newcommand{\tool}{\textsc{CDFix}\xspace}

\newtheorem{Definition}{Definition}
\newtheorem{Claim}{Claim}
\newtheorem{Lemma}{Lemma}
\newtheorem{Theorem}{Theorem}

\newcolumntype{L}[1]{>{\raggedright\arraybackslash}p{#1}}
\newtheorem{observation}{Observation}
\newtheorem{property}{Property}
\newcommand{\code}[1]{{\footnotesize\texttt{#1}}}
\usepackage{amsthm}
 \definecolor{dkgreen}{rgb}{0,0.6,0}
\definecolor{gray}{rgb}{0.5,0.5,0.5}
\definecolor{mauve}{rgb}{0.58,0,0.82}
\lstset{frame=tb,
  language=Java,
  aboveskip=3mm,
  belowskip=3mm,
  showstringspaces=false,
  columns=flexible,
  basicstyle={\small\ttfamily},
  numbers=left,
  numberstyle=\tiny\color{gray},
  keywordstyle=\color{blue},
  commentstyle=\color{dkgreen},
  stringstyle=\color{mauve},
  breaklines=true,
  breakatwhitespace=true,
  tabsize=4
}


%\usepackage{tikz}
%\usetikzlibrary{shapes.arrows}
%\newcommand{\FancyUpArrow}{\begin{tikzpicture}[baseline=-0.3em]
%		\node[single arrow,draw,rotate=90,single arrow head extend=0.1em,inner
%		ysep=0.1em,transform shape,line width=0.03em,top color=green,bottom color=green!50!black] (X){};
%\end{tikzpicture}}

%\def\BibTeX{{\rm B\kern-.05em{\sc i\kern-.025em b}\kern-.08em
%    T\kern-.1667em\lower.7ex\hbox{E}\kern-.125emX}}

\begin{document}

%\title{{\tool}: Hybrid Code Representation Learning for Machine Learning-based Bug Detection Models}

\title{Dual-Task of Context Learning and Code-Transformation Learning to Improve Automated Program Repair}

%Conference Paper Title*\\
%{\footnotesize \textsuperscript{*}Note: Sub-titles are not captured in Xplore and
%should not be used}
%\thanks{Identify applicable funding agency here. If none, delete this.}
%}

%\author{\IEEEauthorblockN{1\textsuperscript{st} Given Name Surname}
%\IEEEauthorblockA{\textit{dept. name of organization (of Aff.)} \\
%\textit{name of organization (of Aff.)}\\
%City, Country \\
%email address or ORCID}
%\and
%\IEEEauthorblockN{2\textsuperscript{nd} Given Name Surname}
%\IEEEauthorblockA{\textit{dept. name of organization (of Aff.)} \\
%\textit{name of organization (of Aff.)}\\
%City, Country \\
%email address or ORCID}
%\and
%\IEEEauthorblockN{3\textsuperscript{rd} Given Name Surname}
%\IEEEauthorblockA{\textit{dept. name of organization (of Aff.)} \\
%\textit{name of organization (of Aff.)}\\
%City, Country \\
%email address or ORCID}
%\and
%\IEEEauthorblockN{4\textsuperscript{th} Given Name Surname}
%\IEEEauthorblockA{\textit{dept. name of organization (of Aff.)} \\
%\textit{name of organization (of Aff.)}\\
%City, Country \\
%email address or ORCID}
%\and
%\IEEEauthorblockN{5\textsuperscript{th} Given Name Surname}
%\IEEEauthorblockA{\textit{dept. name of organization (of Aff.)} \\
%\textit{name of organization (of Aff.)}\\
%City, Country \\
%email address or ORCID}
%\and
%\IEEEauthorblockN{6\textsuperscript{th} Given Name Surname}
%\IEEEauthorblockA{\textit{dept. name of organization (of Aff.)} \\
%\textit{name of organization (of Aff.)}\\
%City, Country \\
%email address or ORCID}
%}

\maketitle

\begin{abstract}
%Recent advances in deep learning (DL) have helped improve the
%performance of the DL-based Automated Program Repair (APR) approaches.
The bug-fixing code changes in Automated Program Repair (APR) often
depend on code context. Despite successes, the state-of-the-art
Deep Learning (DL)-based APR approaches are limited in {\em
  integrating context learning and code-transformation learning},
leading to their ineffectiveness in fixing {\em context-dependent
  bugs}. {\em Learning the correct context for a bug will help
  learning the correct code transformation for the bug fix, and vice
  versa}. In this work, we introduce {\bf \tool}, a context-aware
dual-task learning APR model, that explores the duality of the two
dedicated models to explicitly propagate the mutual impacts of {\em
  context learning} and {\em code transformation learning} onto one
another. We train two models simultaneously with soft-sharing
parameters via a cross-stitch unit for explicit propagation of impacts
to improve both tasks, leading to better APR accuracy.


%Moreover, in those approaches, the cascading architecture from a code
%context-learning model (CCL) to a code-transformation learning model
%(CTL) possibly leads to confounding inaccuracies.
%In this work, we introduce {\bf \tool}, a context-aware dual-task
%learning APR model, that explores the duality of those two dedicated
%models to explicitly propagate the mutual impacts of CCL and CTL onto
%one another. Instead of cascading CCL $\rightarrow$ CTL, we train them
%simultaneously with soft-sharing parameters via a cross-stitch unit
%for explicit propagation of impacts to improve both tasks, leading
%to better APR accuracy.

We conducted several experiments to evaluate {\tool} on three
different datasets: Defects4J~\cite{defects4j} (395 bugs),
Bugs.jar~\cite{saha2018bugs} (1,158 bugs), and
BigFix~\cite{yioopsla19} (+4.9M methods and 1.8M buggy ones).  We
compared {\tool} against several state-of-the-art DL-based APR
tools. Our results show that {\tool} can fix 4.7\%--143\% and
5.7\%--263\% more bugs than the baseline models with only top-1
patches in Bugs.jar and BigFix, respectively.
%In Defects4J, it improves over the baselines from 16.7\%--194.7\%.
In Bugs.jar and BigFix, it fixes 26.4\% and 27.7\% of the total bugs
that were missed by the best DL-based baseline. In Defects4J, it is
complementary with the best baseline Recoder, while improving
2.7\%--200\% over other baselines.
  



\end{abstract}

%In this work, we propose {\tool}, a Context-aware, Graph-based,
%Commit-level Vulnerability Assessment Model that evaluates a
%vulnerability-introducing code and provides the CVSS assessment grades
%for the vulnerability. One of the key solutions is a novel
%context-aware, graph-based, representation learning (RL) model to
%learn the contextualized embeddings for the code changes that
%integrate program dependencies and the surrounding contexts of code
%changes. We utilize the contextualized embeddings to learn to provide
%the CVSS assessment grades for the vulnerability. During the
%assessment of one aspect, we also consider the impacts of the other
%aspects by leveraging a multi-task learning model (each task learning
%to assess one aspect) to propagate the learning from one task to
%another.
%Our empirical evaluation shows that
%on a large dataset of vulnerabilities in C code, {\tool} achieves
%F-score of 25.5\% and MCC of 26.9\% relatively higher than the
%state-of-the-art model in vulnerability assessment. In the dataset of
%vulnerabilities in Java, {\tool} achieves F-score of 14.3\% and
%MCC of 8.0\% relatively higher than the state-of-the-art model in
%vulnerability assessment.

%\begin{IEEEkeywords}
%component, formatting, style, styling, insert
%\end{IEEEkeywords}


\section{Introduction}

Detecting and fixing software defects is crucial for a software
development process. To reduce the efforts from developers in that
process, several {\em fault localization} (FL)
approaches~\cite{fl-survey} have been introduced to help localize the
source of the fault that needs to be fixed. The input of an FL model
is the execution of a test suite, in which some of the test cases are
passing or failing ones. Specifically, the key input is the {\em code
  coverage matrix} in which the rows and columns correspond to the
statements and test cases, respectively.  Each cell is assigned with
the value of 1 if the respective statement is executed in the
respective test case, and with the value of 0, otherwise.  An FL model
uses such information to identify the list of {\em suspicious lines of
  code} that are ranked based on their associated {\em suspiciousness
  scores}~\cite{fl-survey}. In recent advanced FL, several approaches
also support fault localization at method
level~\cite{DeepFL,icse21-fl}. 



%In the FL problem, given the execution of test cases, an FL tool
%identifies the set of {\em suspicious lines of code} with their
%associated suspiciousness scores~\cite{fl-survey}.  The key input of
%an FL tool is the {\em code coverage matrix} in which the rows and
%columns correspond to the source code statements and test cases,
%respectively.  Each cell is assigned with the value of 1 if the
%respective statement is executed in the respective test case, and with
%the value of 0, otherwise. In recent FL, several researchers also
%advocate for fault localization at method level~\cite{DeepFL}. FL at
%both levels are useful for developers.

The FL approaches can be broadly divided into the following
categories: {\em spectrum-based fault localization} (SBFL)
approaches~\cite{Ochiai,jones2001visualization,keller2017critical},
{\em mutation-based fault localization} (MBFL)
approaches~\cite{MUSE,papadakis2012using,Metallaxis}, and {\em machine
  learning (ML)} and {\em deep learning (DL)}~\cite{DeepFL,icse21-fl}.
For SBFL approaches, the key idea is that a line covered more in the
failing test cases than in the passing ones is more suspicious than a
line executed more in the passing ones.
%
To improve SBFL, MBFL
approaches~\cite{MUSE,papadakis2012using,Metallaxis} enhance the code
coverage information by modifying a statement with mutation operators,
and then collecting code coverages when executing the mutated programs
with the test cases. The MBFL approaches apply suspiciousness score
formulas in the same manner as in SBFL approaches on the matrix for
each original statement and its mutated ones.
%
ML and DL-based FL approaches explore the code coverage matrix and
apply different neural network models for fault localization.


%{\em Spectrum-based fault localization} (SBFL)
%approaches~\cite{Ochiai,jones2001visualization,keller2017critical}
%take the recorded lines of code that were covered by each of the given
%test cases, and assigned each line of code a suspiciousness score
%based on the code coverage matrix. Despite using different
%formulas to compute that score, the idea is that a line covered more
%in the failing test cases than in the passing ones is more suspicious
%than a line executed more in the passing ones. A key drawback of those
%approaches is that the same score is given to the lines that have been
%executed in both failing and passing test cases. An example is the
%statements that are part of a block statement and executed at the
%same nested level. Another example is the conditions of the
%condition statements, e.g., \code{if}, \code{while}, \code{do},
%and \code{switch}. 

%To improve SBFL, {\em mutation-based fault localization} (MBFL)
%approaches~\cite{MUSE,papadakis2012using,Metallaxis}
%enhance the code coverage information by modifying a statement with
%mutation operators, and then collecting code coverages when executing
%the mutated programs with the test cases. They apply suspiciousness
%score formulas in the same manner as the spectrum-based FL approaches on
%the code coverage matrix for each original statement and its mutated
%ones. Despite the improvement, MBFL are not effective for the bugs
%that require the fixes that are more complex than a mutation
%(Section~\ref{motivexample}).

%{\em Machine learning (ML)} and {\em deep learning (DL)}
%have been used in fault localization. DeepFL~\cite{DeepFL}
%computes for each faulty method a vector with +200 scores in which
%each score is computed via a specific feature, e.g., a spectrum-based
%or mutation-based formula, or a code complexity metric. Despite its
%success, the accuracy of DeepFL is still limited. A reason could be
%that it uses various calculated scores from different formulas as a
%proxy to learn the suspiciousness of a faulty element, instead of
%fully exploiting the code coverage. Some formulas, such
%as the spectrum- and mutation-based formulas, inherently suffer from
%the issues as explained earlier with the statements covered by
%both failing and passing test cases.

Despite their successes, the state-of-the-art FL approaches still do
not support locating all the fixing locations that need to be repaired
at the same time in the same fix. In real-world software development,
there are several bugs that require a fix to multiple lines of code in
one or multiple hunks in the same or different methods. {\em The
  fixing changes to those lines of code are dependent to one another
  and need to be made in the same fix for the program to pass the test
  cases}. For those bugs, applying the fixing change to one statement
at a time will not make the program pass the test case after the
change to one statement.
%
This capability to detect the fixing locations of the co-changes in a
fix for a bug (let us call it {\em Co-change Fixing Locations} ({\em
  CC Fix Locations})) is important for both the manual process of bug
fixing as well as the automated process of program repairing. For
manual process, such capability will save effort and time for
developers in locating all the buggy statements that need to be fixed
at the same time. For automated program repair (APR), such capability
will enable an APR model to correctly and completely make the changes
to fix a bug.

From the ranked list of suspicious statements returned from an
existing FL model, a naive approach to detect CC Fix Locations would
be to take the top $k$ statements in that list and to consider them as
to be fixed together. This solution might not be effective
because the mechanisms used in the state-of-the-art FL approaches have
never considered the co-change nature of those fixes. Our empirical
evaluation also confirmed that (Section~\ref{eval:sec}).

Detecting all the fixing locations at multiple statements in
potentially multiple methods is challenging. A naive solution would be
detecting the potential methods that need to be fixed together and
then detecting potential statements that need to be changed together
in each of those methods. However, doing so will create a comfounding
effect from the inaccuracy of the detection of buggy methods to the
one of the buggy statements.

{\bf FIXME.} We propose {\tool}, a fault localization approach for
buggy statements/methods ...

The contributions of this paper are listed as follows:

{\bf 1. Novel code coverage representation.} Our representation
enables fully exploiting test coverage matrix and taking advantage of
the CNN model in image recognition to localize~faults.

{\bf 2. {\tool}: Novel DL-based fault localization approach.} Test
case ordering and three sources of information allow treating FL as a
pattern recognition. Without ordering and statement dependencies, the
CNN model will not work~well.

{\bf 3. Extensive empirical evaluation.} We evaluated our model
against the most recent FL models at the statement and method levels,
in both within-project and cross-project settings, and for both C and
Java. Our replication package is available
at~\cite{FaultLocalization2021}.


\section{Motivating Example}
\label{motiv:sec}

\subsection{An Example and Observations}

\begin{figure}[t]
	\centering
	\lstset{
		numbers=left,
		numberstyle= \tiny,
		keywordstyle= \color{blue!70},
		commentstyle= \color{red!50!green!50!blue!50},
		frame=shadowbox,
		rulesepcolor= \color{red!20!green!20!blue!20} ,
		xleftmargin=1.5em,xrightmargin=0em, aboveskip=1em,
		framexleftmargin=1.5em,
                numbersep= 5pt,
		language=Java,
    basicstyle=\scriptsize\ttfamily,
    numberstyle=\scriptsize\ttfamily,
    emphstyle=\bfseries,
                moredelim=**[is][\color{red}]{@}{@},
		escapeinside= {(*@}{@*)}
	}
	\begin{lstlisting}[]
public LegendItemCollection getLegendItems() {
  LegendItemCollection result = new LegendItemCollection();
  if (this.plot == null) {
    return result;
  }
  int index = this.plot.getIndexOf(this);
  CategoryDataset dataset = this.plot.getDataset(index);
  (*@{\color{red}{- if (dataset != null) {}@*)
  (*@{\color{cyan}{+ if (dataset == null) {}@*)
      return result;
  int seriesCount = dataset.getRowCount();
  // update result ...
  return result;
}
	\end{lstlisting}
        \vspace{-9pt}
        \caption{Mutual Impact betw. Context and Change Learning}
        \vspace{-8pt}
        \label{fig:motiv}
\end{figure}

%       if (plot.getRowRenderingOrder().equals(SortOrder.ASCENDING)) {
%            for (int i = 0; i < seriesCount; i++) {
%                if (isSeriesVisibleInLegend(i)) {
%                    LegendItem item = getLegendItem(index, i);
%                    if (item != null) {
%                        result.add(item);
%                    }
%                }
%            }
%        }
%        else {
%            for (int i = seriesCount - 1; i >= 0; i--) {
%                if (isSeriesVisibleInLegend(i)) {
%                    LegendItem item = getLegendItem(index, i);
%                    if (item != null) {
%                        result.add(item);
%                    }
%                }
%            }
%        }

%Let us start with a real-world example to motivate our approach.
Fig.~\ref{fig:motiv} shows a real bug from the project \code{Chart}
in the Defects4J dataset. The bug occurred at line 8 in which the
condition for stopping the updating process on the \code{result}
variable is incorrect (\code{if (dataset != null)}). The result is
returned only when the dataset (line 7) is empty. Thus, it was fixed
into \code{if (dataset == null)}.

\noindent {\bf Observation 1 [Fixing Change Depends on Context]}.~As
seen, the bug-fixing change from line 8 to line 9 (\code{if (dataset
  != null)} $\rightarrow$ \code{if (dataset == null)}) depends on the
code context of the method \code{getLegendItems}. On line 7, the
dataset is retrieved via \code{getDataset}. According to the logic of
the program, \code{result} is returned when \code{dataset} is
\code{null}.  Thus, the incorrect checking was fixed into line 9. That
is, that fix \code{if (dataset == null)} (rather than \code{if
  (dataset == 0)}) makes sense in the context of this method.

%consisting of the preceding code (\code{getDataset},
%\code{getIndexOf}, \code{LegendItemCollection}, etc.), and the
%succeeding code (\code{dataset.getRowCount}, \code{return result},
%etc.).

\vspace{2pt}
\noindent {\bf Observation 2 [Learning Key Contextual Features Depends
    on Fixing Change]}. Let us consider the entire method as a
context. Because the change is (\code{if (dataset} \code{!=}
\code{null)} $\rightarrow$ \code{if (dataset} \code{==} \code{null)}),
the key contextual features are likely the statements at lines 6--7
with \code{getIndexOf} and \code{get\-Dataset}, rather than the other
statements, e.g., at lines 2--5.

%\code{this.plot == null} (line 3) or \code{return result;} (line 4).

Despite successes, the state-of-the-art, DL-based models~are
limited in integrating context learning and change learning.

\underline{First}, for the DL-based approaches that learn the fixes
from prior {\em similar bug fixes or
  patterns}~\cite{gupta2017deepfix,white2019sorting,white2016deep},
there might not be any fix similar to Fig.~\ref{fig:motiv}. They focus
on similar fixes with little or no consideration on whether a fix
appears in certain~context.

\underline{Second}, some other DL-based APR approaches focus {\em only
  on learning the changes} to fix an AST subtree or a
statement~\cite{chakrabortycodit,see2017get} without considering the
context. For this example, without examining the context,
%e.g., the preceding code \code{getDataset} or the succeeding code
%\code{getRowCount} or \code{return result},
such a model will make the same change regardless of contexts.

%not likely learn to change line 8 into line 9.

%hata2018learning,tufano2019learning,tufano2018empirical

\underline{Third}, for the machine translation and transformer-based
APR
models~\cite{hata2018learning,tufano2019learning,tufano2018empirical},
the entire method in Fig.~\ref{fig:motiv} is used as the input. {\em
  Without distinguishing the boundary of the context and the fixing
  changes}, such a model faces the noise.
%, i.e., the code irrelevant to the actual fix at lines 8--9.
For example, the code at lines 2--5 on the legends is not crucial for
the fix regarding the dataset at line 9. Thus, such a model could
identify the incorrect location to fix.

\underline{Fourth}, some other DL-based approaches have separate
representations for
contexts~\cite{chen2018sequencer,cure-icse21,lutellier2020coconut}.
SequenceR~\cite{chen2018sequencer},
CoCoNuT~\cite{lutellier2020coconut}, and CURE~\cite{cure-icse21}
extract features in the surrounding context (the method)
%(e.g., lines 2--7, lines 10--15)
to be fed into a DL model.
%to learn to fix.

%These approaches utilize only one DL model for learning the fixes
%using the features extracted from the contexts.

%\underline{Finally}, recent DL-based APR
%approaches~\cite{icse20,cure-icse21} have leveraged the context to
%help better fix a bug. They separately consider the surrounding code
%as the context (e.g., lines 2--7, lines 10--15).

\underline{Fifth}, a recent trend shows that dedicating a separate
model for context learning can perform better than using a single
model~\cite{icse20}. DLFix~\cite{icse20} has two models: one for
context learning and another for transformation learning.
%in which one tree-based LSTM model learns the context and another one
%learns the code transformation (e.g., from line 8 to line 9).
It cascades the first model to the second one in which the output
of the model for contexts is used as a weight for the transformation
learning one. Thus, DLFix~\cite{icse20} suffers two limitations: 1)
it does not capture well both directions of the mutual impact between
two types of learning to help each other,
%(esp. from code-change learning to context learning),
2) it creates confounding inaccuracies as explained.
%
%This cascading architecture creates a confounding effect from the
%inaccuracy of the learning of the context to the learning of the
%transformations.
In Section~\ref{sec:overlap}, we will present our study and examples
to illustrate this.

%those limitations of the existing DL-based APR approaches.

\subsection{Key Ideas}
\label{sec:key-idea}

%To address the above issue with the cascading architecture,

From the observations, our idea is that to improve APR for {\bf
  context-dependent bugs}, we need {\em both better context learning}
(learning the correct contextual features for a bug fix) and {\em
  better code-transformation learning} (learning correct code
changes). {\tool} treats this problem as a dual-task learning between
CCL and CTL. The impact from both directions (explicitly modeled in
{\tool}) helps both models learning better in its own task, leading to
better APR via CTL.

%Tien
%To advance DL-based APR, we {\bf {\em explicitly model both directions
%    of the mutual impact between context learning and transformation
%    learning}}. We design {\tool} that treats context learning and
%transformation learning as a dual task between CCL and CTL.


%with code context learning model (CCL) and code transformation
%learning (CTL).

%{\tool} consists of two models. The first model, CCL, is dedicated to
%learn \underline{c}ode \underline{c}ontexts, and the second model,
%CTL, to learn bug-fixing \underline{c}ode \underline{t}ransformations.


%To avoid the confounding effect in a naive solution of detecting buggy
%methods first and then detecting buggy statements in those methods, we
%design an approach that treats detecting dependent CC fixing locations
%as a {\em dual learning} task between them. First, the {\em
%  method-level FL} model (\code{MethFL}) aims to learn the methods
%that need to be modified in the same fix. Second, the {\em
%  statement-level FL} model (\code{StmtFL}) aims to learn the
%co-fixing statements regardless of whether they are in the same or
%different methods.

%Tien
%Intuitively, the two models CCL and CTL are dependent on each
%other. The learning of the contexts can benefit the learning of
%bug-fixing code transformations and vice versa.
We refer to this relation as {\em duality}, which can provide useful
constraints for {\tool} to learn to fix {\em context-dependent
  bugs}.
%Tien
%We conjecture that the join training of the two models can
%improve the performance of both, if we can achieve the shared
%representations.
%Tien
%For example, in Figure~\ref{fig:motiv}, if the context is observed as
%containing \code{getDataset} at line 7, \code{getIndexOf} at line 6,
%and \code{return result} at line 10, the likelihood of the fixing
%change at line 8 becoming \code{(dataset == null)} is more than that
%of \code{(dataset != 0)}. The rationale is that only if the retrieved
%data is empty, the result is returned. On the other hand, if the
%bug-fixing transformation is observed as \code{(dataset != null)}
%becoming \code{(dataset == null)}, it is likely to have an assignment
%\code{dataset = ...;} in the context preceding \code{(dataset !=
%  null)}.
The join-training in the dual-task learning makes both models learn
the impact/connections between the correct features in the context
(e.g., the lines 6--7 with \code{getIndexOf} and \code{getDataset})
and the correct code changes (line 8 to line 9). Thus, we
joinly train CTL for context learning and CCL for transformation
learning to help both models make those connections, leading to better
APR for context-dependent~bugs.

%with soft-sharing of parameters to exploit their relation.

{\bf {\em We have also adapted/modified the cross-stitch
    unit \cite{misra2016cross} to work with AST representations}} to
connect CTL and CCL. The sharing of representations between them is
modeled by the learning a linear combination of the input features
from two models. Cross-stitch unit helps regularize both CCL and CTL
by learning and enforcing shared representations by combining feature
maps. This joint training {\em propagates the mutual impact} of
context learning and transformation learning. {\em The use of
  cross-stitch unit also helps avoid confounding inaccuracies} in the
cascading architecture.

%and vice versa, to improve APR performance.

\section{Approach Overview}
\label{overview:sec}

{\tool} has two main processes: training and predicting.

\subsection{Training Process}

Figure~\ref{overview-training} displays the general architecture of
{\tool}'s training process. The input of the training process is the
source code of the buggy method and one buggy statement. If a method
has multiple buggy statements, we treat one buggy statement and that
enclosing method at a time as a training instance. The output includes
the trained tree-based code context learning model (\code{CCL} model
to learn the surrounding code context) and the trained tree-based code
transformation learning model (\code{CTL} model to learn the
bug-fixing code transformation) with their parameters. The training
process has two main steps:

\begin{figure}[t]
	\centering
	\includegraphics[width=3.4in]{graphs/overview-training.png}
	\caption{{\tool}: Training}
	\label{overview-training}
\end{figure}

\noindent {\bf Tree-based Representation Learning.} The goal of this
step is to take the source code under study and to build the
tree-based vector representations (embeddings) to be the input for our
dual models: \code{CCL} and \code{CTL}. To achieve that, we first
parse the given source code to obtain the abstract syntax tree (AST)
for the entire method and the subtree for the buggy statement.  We
then use a word embedding technique to produce the vector for each
node in the AST when we flatten the AST. The output of this step is
the AST for the method and the AST subtree for the buggy statement in
which each node is replaced by its embedding vector
(Figure~\ref{overview-training}).

\noindent {\bf Context-aware Dual Learning Automated Program Repair.}
The goal of this step is to train both of the tree-based context
learning model (\code{CCL}) and the tree-based code transformation
mdeol (\code{CTL}) at the joint training manner. The entire AST of the
buggy method after vectorization (i.e., each node is a vector) is used
at the input layer of the context learning model (\code{CCL}) for
training. The AST of the corresponding fixed method after
vectorization is used at the output layer of the \code{CCL}
model. Similarly, the AST subtree of the buggy statement after
vectorization is used at the input layer of the transformation
learning model (\code{CTL}), and the subtree of the fixed statement
after vectorization is used at the output layer of the \code{CTL}
model. Each of the CCL and CTL models is realized via an
attention-based \code{seq2seq} model~\cite{yi}. Instead of cascading
the two models \code{CCL} and \code{CTL}, we use a mechanism called
{\em cross-stitch unit}~\cite{misra2016cross}, to train them
simultaneously with soft-sharing the parameters to exploit this
duality. The sharing of representations between \code{CCL} amd
\code{CTL} is modeled by learning a linear combination of the input
features in both models. The output of this step includes the trained
\code{CCL} and \code{CTL} models.

\subsection{Prediction Process}

\begin{figure}[t]
	\centering
	\includegraphics[width=3.4in]{graphs/overview-predict.png}
	\caption{{\tool}: Fixing}
	\label{overview-fixing}
\end{figure}


\section{Tree Extraction}

\begin{figure}[t]
	\centering
	\includegraphics[width=3.2in]{graphs/tree_extraction.png}
	\caption{Abstract Syntax Tree Extraction}
	\label{tree-extraction}
\end{figure}


The first step of the \tool is the tree extraction step designed to extract abstract syntax tree (AST) from the source code. It accepts the buggy method with the changed statements inside as input. The output of this step is the extracted AST for the whole buggy method and the extracted sub-tree of AST that represents the changed statements.

Specifically, for a buggy method $m$, \tool uses the Java package JDT \cite{JDT} to generate the AST to represent the buggy method. And for a buggy statement $s$ in the buggy method, \tool uses the sub-tree of AST that exactly covers the statement $s$ to represent the buggy statement. For example, in Figure \ref{tree-extraction}, the AST in the left represents the buggy method in Figure \ref{fig:motiv}, and the right sub-tree of AST in Figure \ref{tree-extraction} represents the buggy statement. If there is more than one buggy statement in the buggy method $m$, \tool generates multiple sub-tree of AST to represent each buggy statement.

Also, when training the model, \tool needs ground truth to let the model learn the parameters, \tool also generates the AST to represent the fixed method $m_f$ and the sub-tree of AST to represent the fixed statement $s_f$. Here, $m_f$ is the after fixing version of buggy method $m$, and $s_f$ is the after fixing version of buggy statement $s$. Thus, \tool uses the AST and sub-tree of AST for fixed method and fixed statement as the ground truth to train the model parameters.

For the buggy method, $m$ and corresponding fixed method $m_f$, \tool can easily find them from the dataset based on the true labels. However, pairing the buggy statement $s$ with its corresponding fixed version $s_f$ is not easy as the methods. To solve this problem, we use an existing approach CPatMiner \cite{nguyen2019graph} to process the fixing changes. Based on the results from CPatMiner, we pair the buggy statement $s$ with the corresponding fixed statement $s_f$ within the three following conditions. 1) If the buggy statement $s$ needs to be deleted, we pair $s$ with an empty statement. 2) if the buggy statement $s$ needs to be updated, we pair $s$ with the updated statement. 3) If there needs to insert a new statement as the fixing, we check the AST for the method $m$ first. And we pair the parent node with the inserted statement $s_f$ if the parent node representing the other statement, or we pair an empty statement with the inserted statement $s_f$. 
\section{Dual-Task Learning Program Repair}
\label{sec: dual-learning}
\begin{figure}[t]
	\centering
	\includegraphics[width=3in]{graphs/cross-stitch}
        \vspace{-9pt}
	\caption{Cross-Stitch Unit for Joint Training~\cite{misra2016cross}}
	\label{fig:cross-stitch}
\end{figure}

After the representation learning step, we obtain the vectorized AST
$T^{M}_b$ for the buggy method $M$ and the vectorized AST subtree
$T^{s}_b$ for the buggy statement $s$, as well as the one for the
respective fixed method $T^{M}_f$ and the one for the respective fixed
statement $T^{s}_f$. $T^{M}_b$ and $T^{M}_f$ are used at the input and
output layers of CCL; and $T^{s}_b$ and $T^{s}_f$ are used at the
input and output layers of CTL (Figure~\ref{fig:dual-learning}).
%Let us explain the dual-task learning framework for CCL and CTL.

%After having the $Tree_m$, $Tree_{mf}$ pair and $Tree_s$, $Tree_{sf}$ pair from the first step, \tool uses them as the input and the ground truth to train the dual learning program repair model in this step. \tool uses the generated after fixing AST $Tree'_m$ and after fixing subtree of AST $Tree'_s$ as output for this step when making the prediction. Specifically, there are two small steps, including the AST node representation learning and the dual learning framework.


\subsection{Cross-Stitch Unit}

We perform a joint training between CCL and CTL via a cross-stitch
unit~\cite{misra2016cross}. Let us first explain the cross-stitch unit
and then how we leverage it for the Automated Program Repair problem.



In Figure~\ref{fig:cross-stitch}, for dual-task learning, we use the
cross-stitch unit to connect CCL and CTL. The sharing of
representations between CCL and CTL is modeled by learning a
linear combination of the input features from the vectorized AST
(sub)trees. The top output of the cross-stitch unit, which becomes
CCL*, gets direct supervision from CCL and indirect supervision from
CTL. Cross-stitch units help regularize both CCL and CTL by learning
and enforcing shared representations by combining feature
maps~\cite{misra2016cross}.

The goal of a cross-stitch unit is to learn a linear combination of
both inputs from CCL and CTL, which are parameterized
using the weights $\alpha$. The output of the cross-stitch unit is
computed as:
\begin{equation}\label{eq:cross-stitch}
	\begin{bmatrix}
		O_C\\
		O_T
	\end{bmatrix}
	=
	\begin{bmatrix}
		\alpha_{CC} &  \alpha_{CT} \\
		\alpha_{TC} &  \alpha_{TT}
	\end{bmatrix}
	\begin{bmatrix}
		I_C\\
		I_T
	\end{bmatrix}
\end{equation}
$I_C$ and $I_T$ are the inputs for the cross-stitch unit, which are
the outputs from a layer of CCL and CTL. $O_C$ and $O_T$ are the
outputs from the cross-stitch unit, which can be used as the inputs
for the next layer of the two models. In this case, the next layers
(noted as CCL* and CTL*) are the ones with the shared representations
and having the joint training. More details on cross-stitch are
in~\cite{misra2016cross}.

\subsection{Tree-based, Dual-Task Learning}

\begin{figure}[t]
	\centering
	\includegraphics[width=2.8in]{graphs/dual-learning-repair-2.png}
        \vspace{-9pt}
	\caption{Dual-Task Learning between CCL and CTL}
	\label{fig:dual-learning}
\end{figure}

We have modified cross-stitch unit to work with AST representations as
follows (Figure~\ref{fig:dual-learning}). We use two separate
attention-based \code{seq2seq} models for CCL and CTL.
%The dual-task training is shown in Figure~\ref{fig:dual-learning}.
We use TreeCaps~\cite{bui2021treecaps} with an attention layer in
between of an encoder and a decoder. In a regular attention-based
\code{seq2seq} model, the hidden state output from the encoder is
directly connected to the attention layer. In {\tool}, we use a
cross-stitch unit to accept as the input the hidden states from the
outputs of the encoders of CCL and CTL. The outputs of the
cross-stitch units are connected to the attention layers of both the
models. This joint training enables propagating the mutual
impact of context learning and transformation learning in both
directions.

\subsubsection*{\bf Formulation.}
As we use TreeCaps~\cite{bui2021treecaps}, which is built on top of
$k$ TBCNN layers~\cite{mou2014tbcnn}, at the $k^{th}$ layer, the
output of the convolution window is calculated as:
\begin{equation}\label{eq:treecaps}
	Y = tanh(\sum_{i=1}^{N}[\Delta^t_iW^t + \Delta^t_iW^t + \Delta^t_iW^t]X_i + b)
\end{equation}
Where $\Delta$ are the weights calculated corresponding to the depth
and the position of the nodes in a tree. This is the mechanism in
TreeCaps to learn the importance of the position of a node in a
tree. $W$ is the trainable matrix; $b$ is the bias factor; $N$ is the
total number of nodes in the convolution window. TreeCaps merges the
output from all TBCNN layers by using a non-linear squash
function~\cite{sabour2017dynamic}. For an AST node $j$, TreeCaps
calculates the capsule $u_j$ as follows (Details of the computation of
$Y$ and $u_j$ can be found in~\cite{bui2021treecaps}):
\begin{equation}\label{eq:2}
	u_j = \frac{||c_j||^2}{||c_j||^2+1}\frac{c_j}{||c_j||}
\end{equation}
We then perform a depth-first-search traversal, and merge all the
capsules of all the nodes in that order. We consider the list of
capsules as the output $U$ of the TreeCaps model. With both
CCL and CTL, we have the outputs $U_C$ for the context
learning part and $U_T$ for the transformation learning part.

With both $U_C$ and $U_T$ as the inputs of the cross-stitch unit, the
outputs of the cross-stitch unit are computed as:
\begin{equation}\label{eq:3}
	\begin{bmatrix}
		X_C\\
		X_T
	\end{bmatrix}
	=
	\begin{bmatrix}
		\alpha_{CC} &  \alpha_{CT} \\
		\alpha_{TC} &  \alpha_{TT}
	\end{bmatrix}
	\begin{bmatrix}
		U_C\\
		U_T
	\end{bmatrix}
\end{equation}
Where $\alpha$ is the trainable weight matrix, $X_C$ and $X_T$ are the
outputs of the cross-stitch unit, which are in turn fed into the
attention layers of both models. $X_C$ and $X_T$ contain the
information learned from both context learning and transformation
learning models, which helps achieve the main goal of dual-task learning
to enhance the APR performance. From Formula~\ref{eq:3}, we have
\begin{equation}\label{eq:4}
	X_C = \alpha_{CC}U_C + \alpha_{CT}U_T
\end{equation}
\begin{equation}\label{eq:5}
	X_T = \alpha_{TC}U_C + \alpha_{TT}U_T
\end{equation}
If the $X_C$ and $X_T$ have different sizes, we need to resize them
for consistence. If the size needs to be increased, we use bilinear
interpolation~\cite{bilinear-interpolation} for resizing. If the size
needs to be reduced, we do the center crop on the matrix to match the
required size. Moreover, for efficiency, we also have a size limit for
the fixed tree in which each buggy tree has at most $P+P^2+...+P^Q$
nodes.  $P$ is the max number of children nodes of a node and $Q$ is
the max child node depth. When predicting, if one vector is close
to zero, we consider it as empty, and all of its children nodes are
considered as empty.

%After solving this, here is one last problem the dual learning may
%face. When making the prediction, because we don't know how the tree
%structure changes, \tool needs to have a size limit to control the
%fixing. \tool expands the child node number to $P$ and expands the
%child node depth to $Q$ for a buggy node. It means that we make each
%buggy node have at most $P+P^2+...+P^Q$ nodes.
%When making predictions, if one vector is close to zero, we consider
%it as empty, and all of its children nodes are considered as empty.

%First, \tool uses two separate attention-based seq2seq frameworks to learn the code fixing for both the method-level and the statement-level. We all use the tree-based deep learning model to process the AST or subtree of AST for the encoder and decoder of these two tasks. Based on the recent study, we select a well-performed baseline TreeCaps \cite{bui2021treecaps} here to do so. Between the encoder and decoder, there is an attention layer for both tasks to help improve the accuracy of generating the fixing.

%In the regular attention-based seq2seq model, the hidden status $H$ is directly passed to the attention layer. However, \tool uses a cross-stitch unit to accept the hidden status $H_m$ and $H_s$ from both method-level and the statement-level to achieve the dual learning. And then, the \tool passes the output of the cross-stitch unit to the method-level and statement-level attention layer. Just like the Figure \ref{program-repair} shown, the output from the encoder does not go to the attention layer. They go to the cross-stitch unit instead. The cross-stitch unit helps both the method-level and the statement-level attention-based seq2seq model catch the input features within the buggy method and the buggy statement.



%As for the TreeCaps model \tool is using, it is built on top of $k$
%TBCNN layers \cite{mou2014tbcnn}. So, for the TBCNN layer $k$, the
%output of the convolution window is calculated as:





%After \tool has $H_m$ and $H_s$ for both method-level and
%statement-level in the encoder, we aim to learn the linear
%combinations of both inputs of the cross-stitch unit. The output of
%the cross-stitch unit is computed as:







%After solving this, here is one last problem the dual learning may face. When making the prediction, because we don't know how the tree structure changes, \tool needs to have a size limit to control the fixing. \tool expands the child node number to $P$ and expands the child node depth to $Q$ for a buggy node. It means that we make each buggy node have at most $P+P^2+...+P^Q$ nodes. When making predictions, if one node is close to zero, we think it is empty and drop it. At the same time, all child nodes of it will be dropped by \tool.

\section{Patch Generation}
\label{sec:patch-gen}

Figure~\ref{overview-fixing} shows the overview of the fixing process.
While the steps of parsing and tree-based representation learning are
the same as in the training process, the step of applying the trained
CCL and CTL models to produce the candidate patches is
different. Let us detail that step, which is illustrated in
Figure~\ref{fig:patch-gen}.

\begin{figure}[t]
	\centering
	\includegraphics[width=3.2in]{graphs/beam-search.png}
	\caption{Patch Generation via Tree-Structured Beam Search}
	\label{fig:patch-gen}
\end{figure}

As explained earlier in Figure~\ref{fig:dual-learning}, the output of the
cross-stitch unit connects to the attention layer, which in turn
connects to the decoder. The decoder has four main components: 1) the
embedding component to encode the input at a time step into the
subtree whose nodes are vectors; 2) the
TreeCaps~\cite{bui2021treecaps} component to summarize the vectors in
a subtree into a vector; 3) the output layer; and 4) the beam search
component, which takes the output at the output layer and produces a
candidate patch (at the current time step $i$) in terms of the output
AST subtree $T_i$ with the concrete source code for the AST
nodes. While the embedding component, TreeCaps, and the output layer
are straightforward, we need to explain the adaptation we have made in
the beam search component.

Beam search is an optimization strategy to keep only the top-$K$ best
solution for the output subtree $T_i$ at a time step $i$ to help
reduce the search space. The reason that we need to use a beam search
strategy is that at each node when we convert a vector back to a code
token, we might have multiple candidate tokens for each node. For
example, in Figure~\ref{fig:patch-gen}, we might have multiple
candidate code tokens for the nodes $N_1$, $N_2$, and $N_3$. Thus, we
might face the combinatorial explosion when considering a potential
large number of nodes in the output subtree $T_i$ at each time step
$i$.

Because the beam search component aims to produce the subtree to be
used as the input of the decoder at the next time step $i+1$
(Figure~\ref{fig:patch-gen}), it must work in accordance with the
TreeCaps component. TreeCaps computes the vector from bottom up (i.e.,
the vectors for children nodes are computed before the one for their
parent node). Therefore, our beam search component considers the order
of the nodes in the AST subtree from the bottom up first, and the
left-to-right order of the sibling nodes of the same parent node. For
example, for the tree $V1$--$V5$ in Figure~\ref{overview-training},
the beam search component will consider the order of the nodes as
follows: $V2$, $V4$, $V5$, $V3$, and $V1$. The sequences of code
tokens in that order for the AST nodes are considered. The score for a
sequence is the product of the probabilities of all the nodes. At each
step, the sequences are ranked and only the top-$K$ best sequences are
maintained.

Another issue occurs during the conversion from an embedding to a code
token. When we search in our dictionary for a token that has a vector
closest to the vector of the current node, we
might encounter a token that is not in the same project or invalid for
the current scope of the program. Thus, we perform static analysis
with several filters to keep only the valid tokens in the current
scope. Specifically, we apply a set of filters to verify the program
semantics as in DLFix~\cite{icse20}. We use the alpha-renaming filter
to change the names back to Java code using a dictionary
containing all the valid names in the scope, the syntax-checking
filter to remove the candidates with syntax errors, and the name
validation filter to check the validity of the variables, methods, and
classes.

Afterward, \tool runs the corresponding test cases for the current
bug. If there is any test case failed, \tool attempts the next
candidate. If all test cases pass, we regards the candidate as
correct.

%The first problem is that when transferring the predicted fixing to
%the real tokens, beam search uses the GloVe embedding
%dictionary. However, this dictionary is learned from multiple
%projects. So the invalid tokens may be generated by the beam
%search. So \tool creates a rule that before using the beam search,
%\tool firstly makes static analysis to select all appeared token in
%the project and when doing the beam search, the results only come from
%the appeared token set.

%Secondly, because the beam search is designed for sequential data,
%\tool makes some small changes to work on the tree structure. To be
%more detailed, \tool regards the nodes with the same parent node as
%the same level. And when doing the beam search, \tool considers these
%nodes at one time by multiply the possibility score for each node. The
%beam search order is from the bottom to the top, which means the node
%with a higher height will be considered first.

%For example, in Figure \ref{patch-validation}, the AST node $N3$ and
%$N4$ will be considered the first when doing beam search. If the
%possibility of $N3=Dataset$ is $p_1$ and the possibility of $N4=null$
%is $p_2$. The first step of beam search will have an optimal token
%like $N3=Dataset, N4=null$ with the possibility of $p_1*p_2$. The
%second step of the beam search is to search $N2$ and $N5$ in the dark
%box simultaneously. And the last step of the beam search is dealing
%with the node $N1$ in the orange box.


%This step will only be used during the prediction, and because the statement-level program repair is our main goal for the dual learning, \tool only consider the output from the statement-level here. After \tool has the predicted fixing for the subtree of AST $Tree_s$, \tool uses the beam search to help the model reduce the search space and run the test cases to do the validation. Beam search is an optimized greedy strategy, and Beam search keeps only the $n$ optimal tokens for each step to help reduce the search space. However, there are two problems the beam search may face when applying to \tool to do the validation.





%After doing the beam search, \tool runs the corresponding test cases for the bug $b_i$. If there is any test case failed, \tool tries the second candidate for the fixing. If all test cases passed, \tool regards the current candidate as the correct fixing for the bug $b_i$, which is the final output for the \tool.



\section{Empirical Evaluation}
\label{sec:eval}

\subsection{Research Questions}

We seek to answer the following research questions:

\noindent\textbf{RQ1. Comparison with State-of-the-art APR Approaches on Defects4J Dataset.}  How well does {\tool} perform compared with the state-of-the-art automate program repair approaches on Defects4J dataset?


\noindent\textbf{RQ2. Comparison with State-of-the-art APR Approaches
  on Large Datasets.}  How does {\tool} perform compared with the
state-of-the-art automate program repair approaches on large datasets?


\noindent\textbf{RQ3. Impact Analysis of Dual-learning Model.} How does the dual learning model affect the overall performance of {\tool}?


\noindent\textbf{RQ4. Evaluation on C Projects.} How does {\tool} perform on C projects?

\subsection{Empirical Methodology}

\subsubsection{Datasets}
We used three datasets that were used in the prior APR research:
Defects4J-v1.2~\cite{defects4j}, Bugs.jar~\cite{saha2018bugs}, and
BigFix~\cite{yioopsla19}. Defects4J-v1.2 has 395 bugs from 6 Java
projects. For each bug in a project, it has the faulty and fixed
versions of the project with test cases. Bugs.jar has 1,158 bugs and
patches from 8 large Java projects. BigFix contains +4.9 million Java
methods, and among them, +1.8 million Java methods are buggy. Bugs.jar
and BigFix contain bug fixes but no test case. We conducted all the
experiments on a server with 16 core CPU and a single Nvidia A100 GPU.

%We perform our valuation on three datasets that have been used
%in the prior research in APR~\cite{icse20}:
%Defects4J~\cite{defects4j}, Bugs.jar~\cite{saha2018bugs}, and
%BigFix~\cite{yioopsla19}. The version of the Defects4J dataset we used
%in this study is the V1.2.0~\cite{defects4j} with 395 bugs
%from 6 Java projects. For each bug in a project, Defects4J has the
%faulty and fixed versions of the project. There are relevant test
%cases for each bug. With the \code{Diff} comparison between faulty and
%fixed versions of a project, we can identify the buggy statements. The
%Bugs.jar dataset contains 1,158 bugs and patches from 8 large, popular
%open-source Java projects. The BigFix dataset contains +4.9 million
%Java methods, and among them, +1.8 million Java methods are
%buggy. There are also the corresponding bug fixes for the buggy
%methods in each dataset as in Defects4J. We conducted all the
%experiments on a server with 16 core CPU and a single Nvidia A100 GPU.

\subsubsection{Evaluation Metrics}

We use the following three metrics:

%to evaluate the performance of \tool and the baseline models.

{\bf 1. Correct Patches/Plausible Patches:} Correct patches are the
fixes that exactly match or have the same semantic meaning as the
developers' fixes in the oracle. For Defects4J (a small dataset), we
manually checked semantic equivalence, however, for Bugs.jar and
BigFix, we used the criteria of exact-matching with the actual
fixes in the oracle.
%
Plausible patches are the fixes that pass all the test cases, but
might not match exactly with the actual fixes.

%It may contain the correct patches and the situation that
%the generated fixing is not correct but passes all test cases.
{\bf 2. P\%:} is the percentage of the generated plausible patches
that exactly match with the actual fixes.
%to be correct ones matching the ground truth by real developers.

{\bf 3. Top-$K$:} is the percentage of the total bugs in which a correct
patch for a bug is in the ranked list of top-$K$ candidate patches.

\subsubsection{Evaluation Methodology.\\}
%We use the following settings for different RQs:

{\bf RQ1. Comparison with DL-based APR Approaches on Defects4J.}
%\underline{Baselines.}
We compare {\tool} with the following
DL-based baselines:

%{\it Hercules \cite{hercules-icse19}: } is a novel APR technique that generalizes single-hunk repair to encompass a specific but significant class of multi-hunk repair problems.

%{\it Tbar \cite{tbar-issta19}: } is a template-based APR to build comprehensive% knowledge about the effectiveness of fix patterns.

{\bf SequenceR~\cite{chen2018sequencer}: } uses the machine
translation approach with sequence-to-sequence learning.

{\bf CoCoNuT~\cite{lutellier2020coconut}:} uses a context-aware neural
machine translation architecture to represent the buggy code
and its context separately.
%However, it does not have the dual learning for context learning and
%code transformation learning as in {\tool}.
It uses ensemble learning on the combination of convolutional neural
networks and a context-aware machine translation.

{\bf DLFix~\cite{icse20}:} DLFix has key differences with {\tool}.
First, CCL and CTL are different. DLFix uses code summarization to
collapse the
%buggy and fixed
subtrees, and uses the new trees with summarized nodes to train
CCL. Second, {\em CCL cascadingly connects to CTL} in which the
vector for the summarized node is used as a {\em weight} representing the
impact from CCL to CTL. The weight vector is used in a cross-product
with all the vectors of the nodes in the subtrees to train CTL. In
{\tool}, dual-task learning is used between CCL and CTL.

%buggy and fixed subtrees to train the CTL model.

%is a two-tier DL-based model that treats APR as code transformation
%learning from the prior bug fixes and the surrounding code
%contexts. It follows a cascading architecture between two models of
%context learning and transformation learning.

{\bf CURE~\cite{cure-icse21}: } is a machine-translation-based program
repair technique using GPT~\cite{radford2018improving} that parses,
models, and searches source code, as opposed to natural language text,
to fix bugs automatically. It treats context separately from the buggy
statements.
%as in CoCoNuT.

%It does not have the dual learning for context learning and code
%transformation learning as in {\tool}.


%Tien
%In this RQ, for %Hercules and
%Tbar, we directly use the results reported in their original paper because they are pattern-based approaches.

For all approaches under study in this RQ, we used the BigFix as the
training dataset and evaluated on Defects4J dataset. 
%For each bug in Defects4J, we used the remaining bugs in Defects4J as the developing dataset to fine-tune the model, and used the fine-tuned model to predict the fix for the bug. 
We use two additional steps for all approaches in comparison:

(1) Fault localization (FL): Conceptually, any FL
tools can be used to produce an ordered list of suspicious
statements that require fixes. We chose Ochiai
algorithm~\cite{abreu2006evaluation, pearson2017evaluating}, which has
been widely used in
APR~\cite{jiang2018shaping,xiong2017precise,koyuncu2018fixminer,xin2017leveraging,wen2018context,liu2018lsrepair}.
After Ochiai localizes a buggy line, all of the AST nodes including
intermediate ones that are labeled by the parser with that buggy line
are collected into a buggy subtree.

(2) Patch validation: Once {\tool} generates a ranked list of
candidate patches, we use a validation
technique~\cite{saha2017elixir,jiang2018shaping} to validate each
candidate.
%Once a candidate patch pases all available test cases, {\tool} stops
%and reports the candidate patch for manual investigation. We report
%the patches that are exactly matched or semantically equivalent to
%the ground-truth fixes in Defects4J.
We set a 5-hour limit for the validation step for all approaches as in
previous work~\cite{icse20,tbar-issta19}.
%Tien
For CURE~\cite{cure-icse21}, we also ran it on its original setting
with unlimited time, with beam search (beam size of 1,000). It
generates 1,000 candidates and the top 5,000 of them are validated.

%I add the CURE* which is the same setting as their original paper. The
%setting is that using beam search, the beam size is 1000, generates
%10000 candidates, validate on top 5000.)

We also used the correct fixing locations for the models to perform
fixing in our study (i.e., without fault localization).
%and used test cases for validation steps.



\underline{Parameter tuning.} We tuned the baselines with the
parameters that mentioned in their papers. We tuned \tool with autoML
\cite{NNI} for the following parameters: {\em epoch}, {\em batch
  size}, {\em learning rate}, {\em embedding length}, the {\em max
  number of children nodes $P$}, and {\em the max children node depth}
$Q$. Our model was tuned with the following key hyper-parameters to
obtain the best performance: (1) Epoch size (i.e., 100, 200, 300); (2)
Batch size (i.e., 64, 128, 256); (3) Learning rate (i.e., 0.001,
0.003, 0.005, 0.010); (4) Vector length of word representation and its
output (i.e., 150, 200, 250, 300); (5) max number of children nodes
$P$ (i.e. 4, 5, 6); (6) max children node depth $Q$ (i.e. 3, 4, 5).

{\bf RQ2. Comparison with DL-based APR Approaches on
  Large Datasets.} We compare {\tool} with the following baselines:
%\underline{Baselines.} We compare {\tool} with the following
%state-of-the-art DL-based APR baseline models:
{\bf Sequen\-ceR~\cite{chen2018sequencer},
  CoCoNuT~\cite{lutellier2020coconut}, DLFix~\cite{icse20},
  CURE~\cite{cure-icse21}:} as in RQ1.

{\bf CODIT~\cite{chakrabortycodit}:} is a DL-based APR approach
using sequence-to-sequence machine translation model with the
abstractions on tree structures to learn the code transformations for
bug fixing.

{\bf Tufano'19~\cite{tufano2019learning}:} is a DL-based approach
aiming to learn code changes by adopting neural machine translation
with code abstractions and filtering via program analysis.

We did not compare with CODIT~\cite{chakrabortycodit} and
Tufano'19~\cite{tufano2019learning} in RQ1, because they do not have
the validation step in their approaches to run on Defects4J. It is
unfair if we compare them with the other approaches having the
validation step. However, we compared {\tool} with them in RQ2
since the two datasets Bugs.jar and BigFix do not have the
test cases for the bugs. Thus, we ran the fixing step of each
model without patch validation on those two datasets.

%We compared {\tool} with those two baselines in this experiment for
%RQ2 running on two large datasets, in which we directly ran the fixing
%step without the validation. The two large datasets do not contain the
%corresponding test cases for the bugs. Thus, we ran a model
%without the validation step on these two datasets.

%The two new baselines in this RQ, including Tufano 19\' and CODIT are not designed for the Defects4J dataset, and there is not validation step in these two approaches. Therefore, comparing these two approaches to other baselines and \tool with the validation step is not fair. So we did not compare these two approaches as baselines in RQ1.

%As for this RQ, for all baselines and \tool, we directly run the
%fixing step without validation. So it is fair to add these two
%baselines.

%We ran the baselines and \tool on two big datasets in this RQ, including Bugs.jar and BigFix.

In RQ2, as running each model, we randomly splitted a dataset
into 80\%/10\%/10\% for training, tuning (validation), and testing.

%We used the same splitting scheme for all models.

In RQ2, we did not use a fault localization tool because Bugs.jar and
BigFix do not have test cases. We used correct fixing locations.

\underline{Parameter tuning:} the same as in RQ1.

%We tuned the baselines with the
%parameters mentioned in their papers and tuned {\tool} as in RQ1.

%in the same manner as in RQ1 with autoML \cite{NNI}.

{\bf RQ3. Overlapping Analysis.} For a baseline model $M$ and {\tool},
we analyzed the number of bugs that were fixed by {\tool}
and missed by $M$, the number of bugs that were fixed by $M$ and
missed by {\tool}, and the number of bugs that were fixed by
both.


{\bf RQ4. Impact Analysis of Dual Learning on Performance.}
%underline{Baselines.} To study the contributions of dual-learning in
%{\tool},
We compare {\tool} with two of its variants:

(1) \textbf{Transformation-only model:} In this variant,
CCL was removed from {\tool} and only CTL was kept. The result
allows us to access the contribution of context learning.
%the context learning model is removed from {\tool} and only the
%bug-fixing code transformation learning model is kept. The result
%allows us to understand the contributions of the context learning
%model (CCL).

%for training in step 2 of \tool.

(2) \textbf{Cascading model:} We also built another variant model in
which we removed the cross-stitch unit for dual-task learning. CCL is
connected to CTL in a cascading manner.
%in which the output of CCL corresponding to a buggy subtree is
%directly used as the input of CTL.
Note: this cascading model differs from DLFix~\cite{icse20} (explained
in RQ1). First, CCL and CTL are different from those in DLFix, which
uses a code summarization technique. Second, in the cascading model,
the output of CCL corresponding to a buggy subtree is directly used as
the input of CTL. In DLFix~\cite{icse20}, the summarized vector is
used in a cross-product to represent the impact from CCL to CTL.

%. In Cascading_Model (Figure_5), part of the CCL output corresponding
%to the buggy subtree is used directly as input to CTL.
%We connected the context learning model (CCL) to the transformation
%learning model (CTL) in a cascading manner as in DLFix~\cite{icse20}
%(the context learning result is added as an additional input of the
%transformation learning model).

%the other naive model, the
%two-tier model, as the baseline. As for the two-tier model, we removed
%the dual-learning from {\tool} and make the statement-level program
%repair is dependent on the output of the method-level program repair.

In this RQ, we used the same process and parameter tuning as in the
experiments for the other RQs. We run all models on BigFix.

%the baselines and \tool in this RQ. Therefore, the parameters that
%eed to be tuned in both baselines are the same as the ones of \tool.

{\bf RQ5. Evaluation on C/C++ Projects.}  To evaluate {\tool} on C/C++
code, we ran it on the C/C++ benchmark
Codeflaws~\cite{tan2017codeflaws} with 3,902 bugs. 
We used the same process and setting as in RQ2.


\section{Experiment Results}

\subsection{\bf RQ1. Comparison with DL-based APR Approaches on Defects4J Dataset.}

Table~\ref{RQ1_defect4j} shows that {\em {\tool} can auto-fix more bugs
than any studied deep-learning-based baselines}. Specifically, {\tool}
can automatically fix 56 bugs and it fixes {\bf 194.7\% (i.e., 37), 40\%
(i.e., 16), 27.3\% (i.e., 12), and 16.7\% (i.e., 8)} more bugs than the
baseline models SequenceR, DLFix, CoCoNuT, and CURE,
respectively. Furthermore, {\tool} generates {\em the most plausible
patches (i.e., 96)} passing all test cases than any other baselines,
indicating that {\tool} has a better patch generation
capability. Moreover, {\tool} has a {\em higher percentage of the generated
plausible patches to be correct than all the baselines}, except
SequenceR. However, {\tool} can fix 194\% more bugs and 6 times more
plausible patches than SequenceR.






%{\footnotesize{
\begin{table}[t]
  \caption{RQ1. Comparison with DL-based APR Approaches on Defect4J.}
  \vspace{-6pt}
  {\small
			\begin{center}
				\renewcommand{\arraystretch}{1}
				\begin{tabular}{p{0.9cm}<{\centering}|p{1.4cm}<{\centering}|p{1cm}<{\centering}|p{1cm}<{\centering}|p{1cm}<{\centering}|p{1cm}<{\centering}}
					
					\hline
					&\textbf{SequenceR}&\textbf{DLFix}& \textbf{Coconut}&\textbf{CURE}&\textbf{\tool}\\
					\hline
					Chart  & 4/5   & 7/13  & 8/13  & 7/12   & 9/12\\
					Closure& 5/7   & 7/12  & 7/18  & 9/27   & 12/24\\
					Lang   & 2/2   & 6/15  & 6/16  & 7/12   & 9/16\\
					Math    & 8/11  & 18/28 & 20/31 & 21/33  & 22/34\\
					Mockito & 0/0   & 1/1   & 2/3   & 2/3    & 2/4\\
					Time    & 0/0   & 1/3   & 2/4   & 3/5    & 3/6\\
					\hline
					Total   & 19/25 & 40/72 & 44/85 & 48/92  & 56/96\\
					\hline
					P(\%)  & 76.0  & 55.6  & 51.8  & 52.2   & 58.3\\
					\hline
				\end{tabular}
			{\footnotesize{
				Note: P is the probability of the generated plausible patches to be correct.\\
				In the cells, x/y: x means the number of correct fixes and y means the number of candidate patches that can pass all test cases. For example, for \tool, 96 candidate patches can pass all test cases. However, 56 out of 96 are the correct fixes compared with the fixes by developers in the ground truth.}}
				\label{RQ1_defect4j}
			\end{center}
                }
		\end{table}
%}}


















%========================================end ===========================


\iffalse
{\footnotesize{
		\begin{table}[t]
			\caption{RQ1. Comparison with the Pattern-based APR Baselines on Defect4J.}
			\begin{center}
				\renewcommand{\arraystretch}{1}
				\begin{tabular}{p{0.8cm}<{\centering}|p{0.6cm}<{\centering}|p{1.1cm}<{\centering}|p{0.8cm}<{\centering}|p{1cm}<{\centering}|p{0.6cm}<{\centering}|p{0.8cm}<{\centering}}
					
					\hline
					&\textbf{Tbar}&\textbf{SequenceR}&\textbf{DLFix}& \textbf{Coconut}&\textbf{CURE}&\textbf{\tool}\\
					\hline
					Chart  & 11/13  & 4/5   & 7/13  & 8/13  & 7/12   & 9/12\\
					Closure& 17/26  & 5/7   & 7/12  & 7/18  & 9/27   & 12/24\\
					Lang   & 13/18  & 2/2   & 6/15  & 6/16  & 7/12   & 9/16\\
					Math   & 22/35  & 8/11  & 18/28 & 20/31 & 21/33  & 22/34\\
					Mockito& 3/3    & 0/0   & 1/1   & 2/3   & 2/3    & 2/4\\
					Time   & 3/6    & 0/0   & 1/3   & 2/4   & 3/5    & 3/6\\
					\hline
					Total  & 69/101 & 19/25 & 40/72 & 44/85 & 48/92  & 56/96\\
					\hline
					P(\%)  & 68.3   & 76.0  & 55.6  & 51.8  & 52.2   & 58.3\\
					\hline
				\end{tabular}
				Note: P is the probability of the generated plausible patches to be correct.\\
				In the cells, x/y: x means the number of correct fixes and y means the number of candidate patches that can pass all test cases. For example, for \tool, 96 candidate patches can pass all test cases. However, 56 out of 96 are the correct fixes compared with the fixes in the ground truth.
				\label{RQ1_defect4j}
			\end{center}
		\end{table}
}}
\fi

\subsection{\bf RQ2. Comparison Results with DL-based APR Approaches on Large Datasets}
\label{rq2:sec}

\begin{table}[t]
	\caption{RQ2. Comparison Results with DL-based APR Approaches on Large Datasets using Top-$K$.}
	\vspace{-10pt}
        {\small
	\begin{center}
		\renewcommand{\arraystretch}{1}
		\begin{tabular}{p{1.6cm}|p{0.7cm}|p{0.7cm}|p{0.7cm}|p{0.7cm}|p{0.7cm}|p{0.7cm}}\hline
			\multirow{2}{*}{Approach}&\multicolumn{3}{c|}{Bugs.jar (1,158 Bugs)}&\multicolumn{3}{c}{BigFix (2,176 Bugs)}\\\cline{2-7}
		                          & Top1   & Top5   & Top10  & Top1   & Top5   & Top10\\
			\hline
			\textbf{CODIT}        & 7.4\%  & 10.3\% & 12.5\% & 7.8\%  & 8.5\%  & 9.2\%\\
			\textbf{Tufano'19}  & 6.5\%  & 9.7\%  & 11.6\% & 4.1\%  & 6.5\%  & 9.4\%\\
			\textbf{SequenceR}    & 8.8\%  & 10.8\% & 12.9\% & 8.3\%  & 9.2\%  & 10.3\%\\
			\textbf{DLFix}        & 10.7\% & 12.1\% & 14.6\% & 11.3\% & 11.8\% & 12.7\%\\
			\textbf{CoCoNuT}      & 12.1\% & 14.2\% & 16.5\% & 12.4\% & 13.5\% & 14.1\%\\
			\textbf{CURE}         & 13.2\% & 14.9\% & 17.4\% & 13.0\% & 13.8\% & 14.5\%\\
                        \textbf{Recoder}         & 14.1\% & 16.1\% & 17.6\% & 14.0\% & 15.2\% & 15.9\%\\
                        \textbf{DEAR}         & 15.1\% & 16.8\% & 18.5\% & 14.1\% & 16.3\% & 17.0\%\\
			\hline
			\textbf{\tool}        & \textbf{15.8\%} & \textbf{18.1\%} & \textbf{19.5\%} & \textbf{14.9\%} & \textbf{17.1\%} & \textbf{18.8\%}\\
			\hline
		\end{tabular}
		\label{RQ2_results}
	\end{center}
        }
\end{table}


As seen in Table~\ref{RQ2_results}, {\tool} can auto-fix more bugs
than any baseline in any metric on both large datasets.  Particularly,
{\tool} can fix 15.8\% of 1,158 bugs in Bugs.jar and 14.9\% of 2,176
bugs in BigFix using only top-1 candidates.

Compared with CURE, {\tool} fixed relatively 12.1\% and 14.6\% more
bugs using only top-1 candidates, and 14.8\% and 16.7\% more bugs using
the top-5 candidates,
%, and 6.3\% and 15.9\% more bugs using the top-10 candidates,
on Bugs.jar and BigFix, respectively. Note: these two datasets have no
test cases, thus, there is no 5-hour limit validation and no CURE*.

Compared with Recoder, {\tool} relatively improves 5\% in Top-1
accuracy in Bugs.jar. With top-1 candidates, it fixed 48 more bugs
(multi-hunk/multi-statement bugs) that Recoder missed, and it missed
40 single-hunk bugs that Recoder fixed. The reason on why {\tool}
performed better than Recoder in Bugs.jar than in Defects4J is that
Recoder does not fix multi-hunk bugs and there is a higher percentage
of multi-hunk/multi-statement bugs in Bugs.jar than Defects4J. The
relative improvements for {\tool} over the baselines in BigFix are
similar to those in Bugs.jar.

\subsection{\bf RQ3. Overlapping Analysis}
\label{sec:overlap}

{\footnotesize{
\definecolor{mygray}{gray}{.9}
\begin{table}[t]
  \caption{RQ3. Overlapping Analysis. The number in the gray boxes: the unique bugs that the current approach can fix. P: \% of the bugs fixed by a tool and missed by the other.}
  \vspace{-9pt}
	\begin{center}
		\renewcommand{\arraystretch}{1}
		\begin{tabular}{p{1cm}<{\centering}|p{1.1cm}<{\centering}|p{0.8cm}<{\centering}|p{0.7cm}<{\centering}|p{1.1cm}<{\centering}|p{0.8cm}<{\centering}|p{0.7cm}<{\centering}}\hline
			Tool &\multicolumn{3}{c|}{Bugs.jar (1,158 Bugs)}&\multicolumn{3}{c}{BigFix (2,176 Bugs)}\\
			\hline
			{\bf CODIT}             & CODIT   & Overlap   & \tool  & CODIT   & Overlap   & {\tool} \\
			\hline
			Fixed \#     & \cellcolor{mygray} 6  & 80   & \cellcolor{mygray} 91  & \cellcolor{mygray} 13 &  157  & \cellcolor{mygray} 165 \\
			P            & 6.8\%   &    & 53.4\%  & 7.7\%   &    & 51.4\% \\
			\hline
			{\bf Tufano'19}             & Tufano 19'   & Overlap   & \tool  & Tufano 19'   & Overlap   & \tool \\
			\hline
			Fixed \#     & \cellcolor{mygray} 5  &  71  & \cellcolor{mygray} 101 & \cellcolor{mygray}4 & 85   & \cellcolor{mygray}237 \\
			P            &  6.2\%  &    &  58.8\% &  4.9\%  &    & 73.6\% \\
			\hline
			{\bf SequenceR}             & SequenceR   & Overlap   & \tool  & SequenceR   & Overlap   & \tool \\
			\hline
			Fixed \#     & \cellcolor{mygray} 7  &   95 & \cellcolor{mygray} 76 & \cellcolor{mygray} 23 &  158  & \cellcolor{mygray} 164 \\
			P            &   6.8\% &    & 44.6\%  &   11.2\% &    & 50.9\% \\
			\hline
			{\bf DLFix} & DLFix   & Overlap   & \tool  & DLFix   & Overlap   & \tool \\
			\hline
			Fixed \#     & \cellcolor{mygray}  19 &  105  & \cellcolor{mygray} 66 & \cellcolor{mygray}35 &  211  & \cellcolor{mygray}111 \\
			P            &  15.0\%  &    & 38.5\%  &  14.2\%  &    &  34.5\%\\
			\hline
			{\bf CoCoNuT}             & CoCoNuT   & Overlap   & \tool  & CoCoNuT   & Overlap   & \tool \\
			\hline
			Fixed \#     & \cellcolor{mygray} 20  & 120   & \cellcolor{mygray} 51 & \cellcolor{mygray}44 &  226  & \cellcolor{mygray} 96\\
			P            &  14.0\%  &    &  29.7\% &  16.1\%  &    & 29.7\% \\
			\hline
			{\bf CURE}             & CURE   & Overlap   & \tool  & CURE   & Overlap   & \tool \\
			\hline
			Fixed \#     & \cellcolor{mygray} 27  &  126  & \cellcolor{mygray} 45 & \cellcolor{mygray} 50&  233  & \cellcolor{mygray} 89\\
			P            &  17.4\%  &    & 26.4\%  & 17.7\%   &    &  27.7\%\\
			\hline
		\end{tabular}
		\label{RQ3_results}
	\end{center}
\end{table}
}}

%To further study the comparison between {\tool} and the baseline
%models,
For a comparative study, we performed an overlapping analysis on the
results.  In Table~\ref{RQ3_results}, each row section corresponds to
the overlapping result between {\tool} and one baseline in two
datasets.

\subsubsection{{\bf Comparison with CODIT}}

As seen in Table~\ref{RQ3_results}, in Bugs.jar, {\tool} can fix 91
bugs that CODIT missed and CODIT fixed 6 bugs that {\tool} missed,
while both can fix the same 80 bugs. Among the total number of bugs
fixed by either tools in Bugs.jar dataset, 53.4\% of them were fixed by
{\tool} and missed by CODIT, while only 6.8\% of them were fixed by
CODIT and missed by {\tool}. In BigFix dataset, while {\tool} can fix
165 bugs that CODIT missed, it missed only 13 bugs that CODIT can fix.



Figure~\ref{example_codit} shows an example that was correctly fixed
by {\tool}, but missed by CODIT. The correct fix is at line 4 by
{\tool}: \code{else} \code{if} \code{(op} \code{instanceof}
\code{ExpressionOperator)}.  However, CODIT fixed it into \code{else}
\code{if(} \code{op)}.
%This is an example that CODIT missed.
CODIT performs patch generation via translation of the buggy AST
sub-tree. It analyzes the buggy sub-tree that covers all the edited
nodes {\em without considering the surrounding context} in the method.
Thus, it did not fix this bug correctly. In this case, {\tool} could
leverage the surrounding context at line 7 with
`\code{currentOp} \code{instanceof} \code{ExpressionOperator}' to
generate the correct patch at line 4.

%{\color{blue}{1. CODIT only do the prediction and token generation on the buggy sub-tree that covers all edited nodes without considering the other information in the same method. So it cannot catch the information about the variable $ExpressionOperator$ in the method.
%2. Fixing results: \textit{CODIT: else if( op )$ \{$} CDFIX: \textit{else if( op instanceof ExpressionOperator ) $\{$}}}

\begin{figure}[t]
	\centering
	\lstset{
		numbers=left,
		numberstyle= \tiny,
		keywordstyle= \color{blue!70},
		commentstyle= \color{red!50!green!50!blue!50},
		frame=shadowbox,
		rulesepcolor= \color{red!20!green!20!blue!20} ,
		xleftmargin=1.5em,xrightmargin=0em, aboveskip=1em,
		framexleftmargin=1.5em,
		numbersep= 5pt,
		language=Java,
		basicstyle=\scriptsize\ttfamily,
		numberstyle=\scriptsize\ttfamily,
		emphstyle=\bfseries,
		moredelim=**[is][\color{red}]{@}{@},
		escapeinside= {(*@}{@*)}
	}
	\begin{lstlisting}[]
public void visit(LOProject project) throws VisitorException {
    ...
(*@{\color{red}{-	\quad else if( op instanceof LOProject ) \{ }}@*)
(*@{\color{cyan}{+ \quad	else if( op instanceof ExpressionOperator ) \{ //correct fix by {\tool} }}@*)
		LogicalExpression expOper = exprOpsMap.get(op);
	...
	if (currentOp instanceof ExpressionOperator) {
		LogicalExpression exp = exprOpsMap.get(currentOp);
	...
  }
// CODIT's incorrect fix at line 3:
// (*@{\color{orange}else if( op)}@*) 
	\end{lstlisting}
        \vspace{-17pt}
	\caption{Comparison with CODIT: An Example}
	\label{example_codit}
\end{figure}

\subsubsection{\bf Comparison with Tufano'19}



\begin{figure}[t]
	\centering
	\lstset{
		numbers=left,
		numberstyle= \tiny,
		keywordstyle= \color{blue!70},
		commentstyle= \color{red!50!green!50!blue!50},
		frame=shadowbox,
		rulesepcolor= \color{red!20!green!20!blue!20} ,
		xleftmargin=1.5em,xrightmargin=0em, aboveskip=1em,
		framexleftmargin=1.5em,
		numbersep= 5pt,
		language=Java,
		basicstyle=\scriptsize\ttfamily,
		numberstyle=\scriptsize\ttfamily,
		emphstyle=\bfseries,
		moredelim=**[is][\color{red}]{@}{@},
		escapeinside= {(*@}{@*)}
	}
	\begin{lstlisting}[]
private Tuple readFromMemory() {
    if (mContents.size() == 0) return null;
    if (mMemoryPtr < mContents.size()) {
(*@{\color{red}{-\quad\quad return ((ArrayList<Tuple>)mContents).get(mMemoryPtr++);}}@*)
(*@{\color{cyan}{+\quad\quad return ((List<Tuple>)mContents).get(mMemoryPtr++);}//correct fix by {\tool} }@*)
	} else {
        return null;
    }
}
// Tufano19's incorrect fix at line 2:
// (*@{\color{orange}if( mMemoryPtr.size () == 0) return null}@*) 
	\end{lstlisting}
        \vspace{-15pt}
	\caption{Comparison with Tufano19': An Example}
	\label{example_tufano19}
\end{figure}

As seen in Table~\ref{RQ3_results}, in Bugs.jar, {\tool} can fix 101
bugs (58.8\%) that Tufano'19 missed and Tufano'19 fixed only 5 bugs
(6.2\%) that {\tool} missed, while both can fix the same 71 bugs.
%Among the total number of bugs fixed by both tools in Bugs.jar
%dataset, 53.4\% of them were fixed by {\tool} and missed by CODIT,
%while only 6\% of them were fixed by CODIT and missed by {\tool}.
In BigFix dataset, while {\tool} can fix 237 bugs that Tufano'19
missed, it missed only 4 bugs that Tufano'19 can fix.

Figure~\ref{example_tufano19} shows an example that was fixed by
{\tool}, but missed by Tufano'19. Tufano'19 uses a {\em
  machine translation approach on the entire buggy method}
\code{readFromMemory}.  Because {\em in training, the model did not
distinguish the buggy statement from the surrounding context (did not
make a clear boundary between them)}, during fixing, the noise from the
irrelevant statements in the method could cause it {\em predict an
  incorrect statement to be fixed}. In this example, Tufano'19
incorrectly identified line 2 as a buggy statement, and fixed it into
\code{if(} \code{mMemoryPtr.size} \code{()} \code{==} \code{0)}
\code{return} \code{null;}. The buggy line 4 was correctly fixed into
line 5 by {\tool}. By distinguishing the context from the fixing
changes, {\tool} could learn the type \code{List} of \code{mContents}
from lines 2 and 3 in which the method \code{size()} is called on
\code{mContents}.

%{\color{blue}{1. Tufano 19' learns and makes prediction only on the method level. In this case, it may fix the wrong position.
%2. Fixing results:
%Tufano 19': \textit{if (mMemoryPtr.size() == 0) return null;} (this fix at line 2)
%CDFIX: \textit{return ((List<Tuple>)mContents).get(mMemoryPtr++);}}}


%=============================================================

\subsubsection{\bf Comparison with SequenceR}

As seen in Table~\ref{RQ3_results}, in Bugs\-.jar dataset, {\tool} can
fix 76 bugs (44.6\%) that SequenceR missed and SequenceR fixed only 7
bugs (6.8\%) that {\tool} missed, while both can fix the same 95 bugs.
%Among the total number of bugs fixed by both tools in Bugs.jar
%dataset, 53.4\% of them were fixed by {\tool} and missed by CODIT,
%while only 6\% of them were fixed by CODIT and missed by {\tool}.
In BigFix, among all the bugs fixed by either
tools, while {\tool} fixed 50.9\% (164 bugs) that SequenceR missed,
it missed only 11.2\% (23 bugs) that SequenceR fixed.

Figure~\ref{example_3} shows an example that SequenceR missed, but was
fixed correctly by {\tool}. Despite considering the context, {\em
  SequenceR treats source code as a sequence of code tokens}. However,
the fix from line 6 into line 7 involves structural changes to the AST
of the statement at line 6 because the order of two operations
\code{abs} and \code{\%} is reversed. In the buggy code, \code{abs} is
executed first and then \code{\%} is next. In the correct code,
\code{\%} is executed before \code{abs}. As shown in
Figure~\ref{tree-change}, the structure of the AST for the statement
is changed to reflect the change in that order. However, SequenceR
considers the statement as the sequence of tokens, and it fixed
by replacing the token \code{hashCode} with the token
\code{get}. That is, it incorrectly fixed the line 6 into
\code{return} \code{(Math.abs(} \code{keyTuple.} \code{get())}
\code{\%} \code{totalReducers);}. Thus, its sequence representations
is not well-suited for structural changes.

%SequenceR: \textit{return (Math.abs( keyTuple.get()) \% totalReducers);}

\begin{figure}[t]
	\centering
	\lstset{
		numbers=left,
		numberstyle= \tiny,
		keywordstyle= \color{blue!70},
		commentstyle= \color{red!50!green!50!blue!50},
		frame=shadowbox,
		rulesepcolor= \color{red!20!green!20!blue!20} ,
		xleftmargin=1.5em,xrightmargin=0em, aboveskip=1em,
		framexleftmargin=1.5em,
		numbersep= 5pt,
		language=Java,
		basicstyle=\scriptsize\ttfamily,
		numberstyle=\scriptsize\ttfamily,
		emphstyle=\bfseries,
		moredelim=**[is][\color{red}]{@}{@},
		escapeinside= {(*@}{@*)}
	}
	\begin{lstlisting}[]
public int getPartition(PigNullableWritable wrappedKey, Writable value, int numPartitions) {
  ....
  indexes = reducerMap.get(keyTuple);
  // if reducerMap does not contain the key, default hash based partitioning
  if (indexes == null) {
(*@{\color{red}{-\quad return (Math.abs(keyTuple.hashCode()) \% totalReducers);	}}@*)
(*@{\color{cyan}{+\quad return (Math.abs(keyTuple.hashCode() \% totalReducers)); // correct fix by {\tool}}}@*)
  }
  ....
}
// SequenceR's incorrect fix at line 6:
// (*@{\color{orange}return (Math.abs( keyTuple. get()) \% totalReducers);}@*)
// CoCoNuT's incorrect fix at line 6:
// (*@{\color{orange}return ( Math.abs( keyTuple. hashCode());}@*)
// CURE's incorrect fix at line 6:
// (*@{\color{orange}return ( Math.abs( keyTuple. hashCode()) / totalReducers);}@*)
// DLFix's incorrect fix at line 6:
// (*@{\color{orange}return ( Math.abs( keyTuple. hashCode()) \% curIndex);}@*)
	\end{lstlisting}
        \vspace{-15pt}
	\caption{Comparison with the Context-aware, DL-based APR baselines: SequenceR, CoCoNuT, CURE, and DLFix}
	\label{example_3}
\end{figure}


\begin{figure}[t]
	\centering
	\includegraphics[width=3.3in]{graphs/example_3.png}
        \vspace{-9pt}
	\caption{AST Structural Changes for the Fix in Figure~\ref{example_3}}
	\label{tree-change}
\end{figure}


\subsubsection{\bf Comparison with CoCoNuT}

As seen in Table~\ref{RQ3_results}, in Bugs\-.jar, {\tool} fixes 51
bugs that CoCoNuT missed and CoCoNuT fixed only 20 bugs that {\tool}
missed, while both fix 120 bugs. Among the total
number of bugs fixed by either tools in BigFix, 29.7\% of them (96)
were fixed by {\tool} and missed by CoCoNuT, while only 16.1\% of them (44)
were fixed by CoCoNuT and missed by {\tool}.

CoCoNuT did not fix correctly the code in Figure~\ref{example_3}.  It
fixed line 6 into \code{return (} \code{Math.abs(} \code{keyTuple.}
\code{hashCode());}. First, CoCoNuT represents the source code by a
sequence of tokens, which is not well-suited for this structural
change in this example. Second, despite considering the context,
CoCoNuT extracts from the surrounding code the features and feeds them
in a single model for learning to fix. {\tool} dedicates a model for
context learning separately from the model for code transformation
learning. In this example, {\tool} is able to learn the structural
changes in the AST (Figure~\ref{tree-change}).

\subsubsection{\bf Comparison with CURE}

As seen in Table~\ref{RQ3_results}, in Bugs.jar, {\tool} can fix 45
bugs that CURE missed and CURE fixed 27 bugs that {\tool} missed,
while both can fix the same 126 bugs. In BigFix dataset, the number of
unique bugs that were fixed by {\tool} but missed by CURE is 178\%
more than the ones that were fixed by CURE but missed by {\tool} (89
versus 50).

CURE did not fix correctly the example in Figure~\ref{example_3}. It
fixed the line 6 into \code{return (} \code{Math.abs(}
\code{keyTuple.} \code{hashCode())} \code{/} \code{totalReducers);}.
That is, it just replaced the operator \code{\%} with \code{/}.
Compared to CoCoNuT, CURE also represents source code as a sequence of
tokens, thus, is not well-suited to learn the structural changes for
the fix in this example. CURE is code-aware,
%however, similar to CoCoNuT, it still encodes
and extracts the code-aware features from the context and feeds them
into a single model. {\tool} treats context learning with the CCL
model separately from transformation learning with the CTL model.
The dual learning helps propagate the impact of both models on each other.

\subsubsection{\bf Comparison with DLFix}

As seen in Table~\ref{RQ3_results}, in Bugs.jar, {\tool} can fix 66
bugs that DLFix missed and DLFix fixed 19 bugs that {\tool}
missed, while both can fix the same 105 bugs. Among the total
number of bugs fixed by both tools in BigFix dataset, 34.5\% of them
were fixed by {\tool} and missed by DLFix, while only 14.2\% of them
were fixed by DLFix and missed by {\tool}.

DLFix also did not fix correctly the buggy code in
Figure~\ref{example_3}. It fixed the line 6 into \code{return (}
\code{Math.abs(} \code{keyTuple.} \code{hashCode())} \code{\%}
\code{curIndex);}. Despite that DLFix learns tree-structured code
transformations with separate models for context learning and
transformation learning, it follows a cascading architecture between
the two models for APR. Thus, the incorrect learning of context might
affect the learning of transformations. In this example,
\code{curIndex} was incorrectly used in the fix. In {\tool}, context
learning and transformation learning are treated as a dual task to
propagate the impact on each other.


%{\color{blue}{1. SequenceR, CoCoNut, CURE all using sequence structure to do the code fixing. However, it is hard for them to learn the structure changes in this example. DLFix learns the structure changes, but it does not learn it as well as CDFix.
%2. Fixing results:
%SequenceR: \textit{return (Math.abs( keyTuple.get()) \% totalReducers);}
%CoCoNut: \textit{return (Math.abs( keyTuple.hashCode()));}

%DLFix: \textit{return (Math.abs( keyTuple.hashCode()) \% curIndex);}

%CURE: \textit{return (Math.abs( keyTuple.hashCode()) / totalReducers);}

%CDFIX: \textit{return (Math.abs( keyTuple.hashCode() \% totalReducers));}}}





%Table~\ref{RQ3_results} shows the overlapping analysis of the bugs fixed by {\tool} and the studied baselines. The results show that {\tool} can fix more unique bugs than any compared baseline. For example, {\tool} can fix 89 bugs that cannot be fixed by CURE, while CURE can fix 50 bugs that our {\tool} cannot fix.


%Consolidating results from RQ2 and RQ3, overall, {\tool} can auto-fix more bugs and unique bugs than any studied baselines, indicating that {\tool} is better and complementary to other baselines.

\subsection{\bf RQ4. Impact of Dual-learning}



\begin{table}[t]
  \caption{RQ4.Impact Analysis Results of Dual Learning using BigFix Dataset.}
  \vspace{-6pt}
	{\small
		\begin{center}
			\renewcommand{\arraystretch}{1}
			\begin{tabular}{p{1cm}<{\centering}|p{2.7cm}<{\centering}|p{1.7cm}<{\centering}|p{1cm}<{\centering}}
				\hline
				Top-$K$ & Transformation-only Model & Cascading Model &  \tool \\			
				\hline
				Top-1   & 7.1\% & 11.7\% & 14.9\% \\ \hline
				Top-5	& 8.9\% & 13.1\% & 16.1\% \\ \hline
				Top-10	& 9.7\% & 14.3\% & 16.8\%\\ \hline
			
				\hline
			\end{tabular}
			\label{fig:rq4_results}
		\end{center}
	}
\end{table}

Table~\ref{fig:rq4_results} shows the result on the impact of our dual
learning architecture on the overall {\tool}'s bug-fixing performance.
As seen, the top-$K$ values of the \code{Transformation-only} model
are 52.3\%, 44.7\% and 42.3\% lower than those of {\tool} in Top-1,
Top-5, and Top-10, respectively. This result shows that 1) the context
learning model in {\tool} has good impact on the overall performance,
and 2) the dual learning enables the impact from context learning to
transformation learning to achieve high performance in APR.

Moreover, the top-$K$ values of the \code{cascading} model are 21.5\%,
18.6\%, and 14.9\% lower than those of {\tool} in Top-1, Top-5, and
Top-10, respectively. This result shows that 1) the cascading
architecture between context learning (CCL) and transformation
learning (CTL) is not effective as the dual-learning architecture as
in {\tool}, and 2) dual learning between CCL and CTL is effective and
helps improve APR performance. This result also explains the reason
for the higher performance of {\tool} over the state-of-the-art APR
approach in DLFix~\cite{icse20}, which has a cascading architecture of
context learning and transformation learning.


%Table~\ref{fig:rq4_results} presents the results of contributions of dual-learning in CDFix. The results show that Only-transformation-model reduces 52.3\%, 44.7\% and 42.3\% of {\tool} using Top-1, Top-5, and Top-10, respectively, which indicates that context-learning model is important to our {\tool}.

%The Cascading model also reduces the Top-1, Top-5, and Top-10 of {\tool} by 21.5\%, 18.6\%, and 14.9\%, respectively, indicating that the simultaneous dual-learning of context learning model and transformation learning model is effective.

\subsection{\bf RQ5. Evaluation on C/C++ Projects}
\label{sec:eval-c}

\begin{table}[t]
	\caption{RQ4. Codeflaws (C/C++ Projects) versus Defects4J (Java Projects). P\% = $|$Fixed Bugs$|$/\{Plausible Patches\}.}
	\vspace{-5pt}
	{\footnotesize
		\begin{center}
			%\renewcommand{\arraystretch}{1}
			\tabcolsep 2.7pt
			\begin{tabular}{p{0.5cm}<{\centering}|p{1.5cm}<{\centering}|p{1.55cm}<{\centering}|p{0.5cm}<{\centering}|p{1.5cm}<{\centering}|p{1.55cm}<{\centering}}\hline	
				%	Metric& BugsInPy (Python projects) &  Defects4J (Java projects)\\\hline
				
				\multicolumn{3}{c|}{Codeflaws } & \multicolumn{3}{c}{Defects4J}\\\hline
				 P\%& \# of Fixed Bugs& \# of Plausible Patches &P\%& \# of Fixed Bugs & \# of Plausible Patches \\ \hline
				
				  76.3  &        513       &            672             &  58.3 &          56        &        96                 \\
				\hline

				%\vspace{1pt}
			\end{tabular}
			\label{RQ5}
		\end{center}
	}
%	\vspace{-10pt}
\end{table}
  
To evaluate {\tool} on the projects with a different programming
language, we ran it on Codeflaws, a dataset of bugs in C/C++ projects.
As seen in Table~\ref{RQ5}, {\tool} is able to correctly fix 513 bugs
in the total of 3,902 bugs in that dataset, with 672 plausible
patches. The percentage of the number of fixed bugs over the plausible
patches for C/C++ projects is higher than that for Java projects
(76.3\% versus 58.3\%). However, the percentage of the fixed bugs in
the total bugs in the each dataset is similar (13.1\% in Codeflaws
versus 14.2\% in Defects4J). In brief, the performance of {\tool} for
C/C++ projects in Codeflaws is consistent with that for Java projects
in Defects4J. This result also shows that {\tool} is able to work on
any programming language because the concepts used in {\tool} such as
code tokens, AST, dual learning, and sub-models are
language-independent.



%As seen in Table~\ref{RQ5}, {\tool} can autofix xx bugs in C/C++
%projects of Codeflaws.  The empirical results show that the
%performance of {\tool} on the C/C++ projects is consistent with the
%one on the Java projects. Specifically, the percentages of the C/C++
%and Java bugs that can be fixed are similar, e.g., 13.1\% vs. 14.2\%,
%respectively.





\subsection{Threats to Validity}
The threats of this paper come from the following aspects: 
(1) We evaluated our {\tool} on Java and C/C++ code. 
The key modules in {\tool} are language-independent and can be applied to
other types of programming languages. 
(2) We re-implemented CURE as its code was publicly available online and also the authors of CURE never replied our emails. We tried our best to follow the details in their paper during the implementation. 
%(3) 


%The compared baselines and FixLocator require a dataset with test cases for generating code coverage information, run-time information, and failing traces, thus we only compared them on Defects4J. In the future, we plan to test FixLocator on more types of datasets. 
%(3) The DeepFL is designed for method-level fault localization. To compare it with FixLocator fairly, we use only part of its features that are applicable to statement level fault localization. The other baselines CNNFL and DeepRL4FL can be directly applied on statement-level FL. 
%(4) To compare fairly, we evaluated the baselines using out proposed Hit-N@Set metrics. However, we also tested our FixLocator using the ranking metrics: Hit-N@Top-K. FixLocator is consistently better than all baselines in any metric.
\subsection{Limitations}
\label{sec:limitations}


%{\tool} still has the following limitations.
First, {\tool} does not determine the faulty statements to be fixed.
%As with some existing work~\cite{cure-icse21,lutellier2020coconut},
A common usage of {\tool} is that one uses a fault localization tool
to detect the faulty statements or (s)he can pinpoint the faulty one
and invoke {\tool} for auto-fixing. Second, it cannot produce a fix
with an arbitrarily large size because its CTL model does not learn
well large tree-structured changes. Third, it cannot fix a bug with
only insertions of new statements, and it does not work well with a
fix with only deletions.
%because it sets a limit with $P$ and $Q$ in the size of the fixed
%subtree.
%It cannot fix multi-statement bugs where the changes depend on one
%another.
%
Finally, {\tool} learns only the tree-based code transformations for
APR. The changes in program semantics such as changes to PDGs, CFGs,
etc. could help improve APR performance in our model.


\section{Related Work}
We summarize the related studies on automated program repair.

{\bf Fixing pattern based APRs.} 
%In the earlier stage, the automated program repair (APR) approaches aimed to automatically derive
%{\em the fixes for similar code} cloned from one place to
%another~\cite{icse10}, or similar code due to porting or
%branching~\cite{ray-fse12}. 
%
Toward addressing the more general
defects, several researchers have explored the {\em search-based
	approaches}~\cite{le2011genprog,qi2014strength,LeGoues-icse12,martinez2016astor}
in which a search strategy is performed in the space of potential
solutions produced by several operators mutating the buggy
code. Then, test cases and/or program verification are applied to
select the better candidate fixes~\cite{smith2015cure}. 
%
In contrast to the search-based approaches, 
the automatic or semi-automatic approaches of {\em mining and learning fixing patterns} from prior bug
fixes~\cite{le2016history, kim2013automatic,nguyen2013semfix,liu2019avatar,tbar-issta19} have been proposed, such as synthesizing a patch using symbolic execution and
constraint solving--SemFix~\cite{nguyen2013semfix},  
learning models from existing submitted patches~\cite{long2016automatic,long2017automatic,le2016history}, synthesizing
patches using method call related patterns--ELIXIR~\cite{saha2017elixir}, mining code change operations (e.g., Insert If- Statement) from the patches in code change histories--CapGen~\cite{wen2018context}, SimFix~\cite{jiang2018shaping} and
FixMiner~\cite{koyuncu2018fixminer}, exploring fix patterns of static analysis violations--Avatar~\cite{liu2019avatar}. Tbar~\cite{tbar-issta19} is a template-based APR tool with the collected fix patterns.
Hercules~\cite{saha2019harnessing} targets at the multi-hunk bugs that may require applying a substantially similar patch to different locations. 
Getafix~\cite{bader2019getafix} learns fix patterns from past fixes and auto-fixes the warnings from static analyzers. 
Our {\tool} is a data-driven and deep learning based approach and different from the above pattern-based tools. 



%Prophet learns a patch ranking model using machine learning algorithm based on existing patches.  
%Genesis~\cite{long2017automatic} can automatically infer patch generation transformed from developers' submitted patches for automated program repair. 
%HDRepair~\cite{le2016history} was proposed to repair bugs by mining closed frequent bug fix patterns from graph-based representations of real bug fixes.
%ELIXIR~\cite{saha2017elixir} uses method call related templates from PAR with local variables, fields, or constants, to construct more expressive repair-expressions that go into synthesizing patches. 

%CapGen~\cite{wen2018context}, SimFix~\cite{jiang2018shaping}, FixMiner~\cite{koyuncu2018fixminer} are based on the frequently occurred code change operations (e.g., Insert If- Statement) that are from the patches in code change histories.
%Avatar~\cite{liu2019avatar} exploits fix patterns of static analysis violations as ingredients for patch generation.


%\textbf{Fix Pattern Mining and Learning based APR.}  Another direction
%of APR approaches tend to automatically mine fix patterns or
%templates.  For example, SemFix~\cite{nguyen2013semfix} instead uses
%symbolic execution and constraint solving to synthesize a patch by
%replacing only the right-hand side of assignments or branch
%predicates.  Long and Rinard also proposed a patch generation system,
%Prophet~\cite{long2016automatic}, that learns code correctness models
%from a set of successful human patches. Prophet learns a patch ranking
%model using machine learning algorithm based on existing patches.
%They further proposed a new system, Genesis~\cite{long2017automatic},
%which can automatically infer patch generation transforms from
%developer submitted patches for automated program repair.  Motivated
%by PAR~\cite{kim2013automatic}, more effective automated program
%repair systems have been explored. HDRepair~\cite{le2016history} was
%proposed to repair bugs by mining closed frequent bug fix patterns
%from graph-based representations of real bug fixes.  Nevertheless, its
%fix patterns, except the fix templates from PAR, still limits the code
%change actions at abstract syntax tree (AST) node level, but are not
%specific for some types of bugs.  ELIXIR~\cite{saha2017elixir}
%aggressively uses method call related templates from PAR with local
%variables, fields, or constants, to construct more expressive
%repair-expressions that go into synthesizing patches.  More recently,
%CapGen~\cite{wen2018context}, SimFix~\cite{jiang2018shaping},
%FixMiner~\cite{koyuncu2018fixminer} are further proposed to fix bugs
%automatically based on the frequently occurred code change operations
%(e.g., Insert If- Statement) that are extracted from the patches in
%developer change histories.  Avatar~\cite{liu2019avatar} exploits fix
%patterns of static analysis violations as ingredients for patch
%generation So far however, pattern-based APR approaches focus on
%leveraging patches that developer applied to semantic bugs.

{\em Deep Learning-based APR approaches}. Recently, deep learning (DL)
has been applied to APR for directly generating patches. 
 The first group of DL-based APR approaches leverage the capability of DL models
in {\em learning similar source code for similar fixes}, such as DeepRepair~\cite{white2016deep} and DeepFix~\cite{gupta2017deepfix}. 
%DeepRepair leverages learned code similarities, captured with recursive
%auto-encoders~\cite{white2016deep}, to select the repair ingredients
%from code fragments that are similar to the buggy code.
%DeepFix~\cite{gupta2017deepfix} learns the syntax rules and is
%evaluated on syntax errors.
%uses deep learning to directly generate patches. The precision of this approach depends on the neutral network learned from the training
%set. However, so far this approach is only evaluated
%on syntactic errors.
Some other approaches apply neural machine translation (NMT) onto APR: translating the buggy code to its fixed version, such as Ratchet~\cite{hata2018learning} and Tufano {\em et al.}~\cite{tufano2018empirical}, SequenceR~\cite{chen2018sequencer}, Tufano {\em et al.}~\cite{tufano2019learning}. However, the above early work using NMT for APR did not perform well on the benchmark dataset Defects4J compared with the well-known pattern-based tools. 

Recently, DLFix~\cite{li2020dlfix}

%CODIT~\cite{chakrabortycodit} learns code edits with encoding code structures in
%an NMT model to recommend fixes. 
%The comparison with these NMT-based APR approaches is provided in the introduction. 
%Recently, Tufano {\em	et al.}~\cite{tufano2019learning} learn code changes using sequence-to-sequence NMT with simple code abstractions and keyword replacing. Despite of treating the APR as code transformation learning problem, their approach takes entire method as the context for a bug. Thus, it has too much noise, leading to lower effectiveness than {\tool}. In other words, the treatment of context from {\tool} helps improve over their model.


%DeepRepair~\cite{white2019sorting} is an early attempt to integrate machine learning in a program repair loop. DeepRepair leverages learned code similarities, captured with recursive autoencoders~\cite{white2016deep}, to select repair ingredients from code fragments that are similar to the buggy code.

%SequenceR~\cite{chen2018sequencer} is another Neural Machine Translation (NMT)-based system to learn source code changes based on a seq2seq model and copy mechanism~\cite{see2017get}. DeepRepair uses machine learning to select interesting code, SEQUENCER uses machine
%learning to generate the actual patch. However, as reported by SequenceR, it can only find 14 correct patches. Another approach using NMT, CODIT~\cite{chakrabortycodit}, has been proposed to learn code edits and recommend possible edits. However, CODIT uses tree-based NMT. Although CODIT is not designed for APR, it is tested on Defects4J. As reported, CODIT can suggest 16 correct patches. Compared with our results on the benchmark dataset Defects4J, DLFix can find xx patches.

%Hata et al.~\cite{hata2018learning} proposes, Ratchet, a patch
%generation system using the basic attention-based Encoder and Decoder
%machine translation model.
%Similar to Ratchet, Tufano et al.~\cite{tufano2018empirical} use the
%basic encoder-decoder machine translation model and code abstraction
%to generate patches. Furthermore, Tufano et
%al.~\cite{tufano2019learning} propose a deep learning based approach
%for learning code changes using the techniques including code
%abstraction, keywords replacing and neural machine translation
%model. They claim that their approach can be applied to APR. Through
%empirical comparisons, our approach DLFix outperforms them.

%Compared with the above studies using deep learning, our approach is mainly different in the following ways: (1) we develop a new two-layer Tree-based NMT model that is different from any existing NMT model; (2) we do not just learn the context from the changed lines, but also the code surrounding a fix, which makes our model more powerful on learning; and (3) we combine program analysis techniques, such as alpha-renaming and verifying program semantics.

\section{Conclusion}

The bug-fixing changes in APR often depend on the surrounding code
context. Despite their successes, the state-of-the-art DL-based APR
approaches still have limitations in the integration of code contexts
in learning bug fixes. In this work, we conjecture that correct
learning of contexts can benefit the learning of code transformations
and vice versa in auto-fixing context-dependent bugs. We introduce
{\tool}, a context-aware dual learning APR model, which dedicates one
model to learn the bug-fixing code transformations (CTL) and another
one to learn the corresponding surrounding code contexts
(CCL). Instead of cascading the two models, we train them
simultaneously with soft-sharing their parameters via a cross-stitch
unit to exploit the duality between CCL and CTL. We conducted several
experiments to evaluate {\tool} on three large datasets. Our results
show that {\tool} is able to fix {\bf XX\%} and {\bf XX\%} more bugs
than the best-performance DL-based baseline models, in which several
of them are context-dependent bugs. {\tool} detected {\bf XX} unique
bugs that all the other DL-based APR models missed.



%We show that a simple combination could help improve BD model. Our
%result calls for actions in developing hybrid CRL models for source
%code.

\section{Data Availability}

Code and data are available at~\cite{CDFix2022}.

\newpage

\balance

%\bibliographystyle{plain}
%\bibliographystyle{ACM-Reference-Format}
\bibliographystyle{IEEEtran}

\bibliography{References}

\end{document}

