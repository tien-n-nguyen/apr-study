\subsection{\bf RQ1. Comparison Results with DL-based APR Approaches on Defects4J Dataset}

Table~\ref{RQ1_defect4j} shows that {\em {\tool} can auto-fix more bugs
than any studied deep-learning-based baselines}. Specifically, {\tool}
can automatically fix 56 bugs and it fixes {\bf 194.7\% (i.e., 37), 40\%
(i.e., 16), 27.3\% (i.e., 12), and 16.7\% (i.e., 8)} more bugs than the
baseline models SequenceR, DLFix, CoCoNuT, and CURE,
respectively. Furthermore, {\tool} generates {\em the most plausible
patches (i.e., 96)} passing all test cases than any other baselines,
indicating that {\tool} has a better patch generation
capability. Moreover, {\tool} has a {\em higher percentage of the generated
plausible patches to be correct than all the baselines}, except
SequenceR. However, {\tool} can fix 194\% more bugs and 6 times more
plausible patches than SequenceR.






%{\footnotesize{
\begin{table}[t]
  \caption{RQ1. Comparison Results with DL-based APR Approaches on Defect4J.}
  \vspace{-6pt}
  {\small
			\begin{center}
				\renewcommand{\arraystretch}{1}
				\begin{tabular}{p{0.9cm}<{\centering}|p{1.4cm}<{\centering}|p{1cm}<{\centering}|p{1cm}<{\centering}|p{1cm}<{\centering}|p{1cm}<{\centering}}
					
					\hline
					&\textbf{SequenceR}&\textbf{DLFix}& \textbf{Coconut}&\textbf{CURE}&\textbf{\tool}\\
					\hline
					Chart  & 4/5   & 7/13  & 8/13  & 7/12   & 9/12\\
					Closure& 5/7   & 7/12  & 7/18  & 9/27   & 12/24\\
					Lang   & 2/2   & 6/15  & 6/16  & 7/12   & 9/16\\
					Math    & 8/11  & 18/28 & 20/31 & 21/33  & 22/34\\
					Mockito & 0/0   & 1/1   & 2/3   & 2/3    & 2/4\\
					Time    & 0/0   & 1/3   & 2/4   & 3/5    & 3/6\\
					\hline
					Total   & 19/25 & 40/72 & 44/85 & 48/92  & 56/96\\
					\hline
					P(\%)  & 76.0  & 55.6  & 51.8  & 52.2   & 58.3\\
					\hline
				\end{tabular}
			{\footnotesize{
				Note: P is the probability of the generated plausible patches to be correct.\\
				In the cells, x/y: x means the number of correct fixes and y means the number of candidate patches that can pass all test cases. For example, for \tool, 96 candidate patches can pass all test cases. However, 56 out of 96 are the correct fixes compared with the fixes by developers in the ground truth.}}
				\label{RQ1_defect4j}
			\end{center}
                }
		\end{table}
%}}


















%========================================end ===========================


\iffalse
{\footnotesize{
		\begin{table}[t]
			\caption{RQ1. Comparison with the Pattern-based APR Baselines on Defect4J.}
			\begin{center}
				\renewcommand{\arraystretch}{1}
				\begin{tabular}{p{0.8cm}<{\centering}|p{0.6cm}<{\centering}|p{1.1cm}<{\centering}|p{0.8cm}<{\centering}|p{1cm}<{\centering}|p{0.6cm}<{\centering}|p{0.8cm}<{\centering}}
					
					\hline
					&\textbf{Tbar}&\textbf{SequenceR}&\textbf{DLFix}& \textbf{Coconut}&\textbf{CURE}&\textbf{\tool}\\
					\hline
					Chart  & 11/13  & 4/5   & 7/13  & 8/13  & 7/12   & 9/12\\
					Closure& 17/26  & 5/7   & 7/12  & 7/18  & 9/27   & 12/24\\
					Lang   & 13/18  & 2/2   & 6/15  & 6/16  & 7/12   & 9/16\\
					Math   & 22/35  & 8/11  & 18/28 & 20/31 & 21/33  & 22/34\\
					Mockito& 3/3    & 0/0   & 1/1   & 2/3   & 2/3    & 2/4\\
					Time   & 3/6    & 0/0   & 1/3   & 2/4   & 3/5    & 3/6\\
					\hline
					Total  & 69/101 & 19/25 & 40/72 & 44/85 & 48/92  & 56/96\\
					\hline
					P(\%)  & 68.3   & 76.0  & 55.6  & 51.8  & 52.2   & 58.3\\
					\hline
				\end{tabular}
				Note: P is the probability of the generated plausible patches to be correct.\\
				In the cells, x/y: x means the number of correct fixes and y means the number of candidate patches that can pass all test cases. For example, for \tool, 96 candidate patches can pass all test cases. However, 56 out of 96 are the correct fixes compared with the fixes in the ground truth.
				\label{RQ1_defect4j}
			\end{center}
		\end{table}
}}
\fi
