\subsection{\bf RQ1. Comparison Results with DL-based APR Approaches on Defects4J Dataset}

Table~\ref{RQ1_defect4j} shows that {\em {\tool} can auto-fix more bugs
than any DL-based baselines}. {\tool}
automatically fixes 56 bugs and it fixes {\bf 194.7\% (i.e., 37), 40\%
(i.e., 16), 27.3\% (i.e., 12), and 16.7\% (i.e., 8)} more bugs than the
baseline models SequenceR, DLFix, CoCoNuT, and CURE,
respectively. Furthermore, {\tool} generates {\em the most plausible
patches (i.e., 96)} passing all test cases than any other baselines,
indicating that {\tool} has a better patch generation
capability. Moreover, {\tool} has a {\em higher percentage of the generated
plausible patches to be correct than all the baselines}, except
SequenceR. However, {\tool} can fix 194\% more bugs and 6 times more
plausible patches than SequenceR.
%
 %and CURE fixed 4 bugs that {\tool} missed. 
% {\tool} fixed 12 unique bugs that CURE cannot fix and 3 unique bugs
% that were missed by all other baselines.
Moreover, CDFix fixed 12 bugs that the {\em best baseline} CURE
missed; and CURE fixed 4 bugs that CDFix missed. CDFix fixed 3 bugs
that all the DL-based baselines missed.

Table~\ref{RQ1_defects4J_with_FL} shows the comparative result on
Defects4J with an FL tool. As seen, with the FL tool, the performance
of all models decreases due to the confounding inaccuracy. However,
{\tool} can still fix more bugs than any baseline. It
automatically fixes {\bf XX} bugs, i.e., {\bf XXX.X\% (i.e., XX), XX\%
  (i.e., XX), XX.X\% (i.e., XX), and XX.X\% (i.e., X)} more bugs than
the baselines SequenceR, DLFix, CoCoNuT, and CURE,
respectively. Furthermore, {\tool} generates the most plausible
patches (i.e., {\bf XX}) among all the models.
%indicating that {\tool} has a better patch generation
%capability. Moreover, {\tool}
It also has a higher percentage of the generated plausible patches to
be correct than all the baselines.
%However, {\tool} can fix 194\% more bugs and 6 times more plausible
%patches than SequenceR.
Moreover, CDFix fixed {\bf XX} bugs that the {\em best baseline} CURE
missed; and CURE fixed {\bf XX} bugs that CDFix missed. CDFix fixed
{\bf XX} bugs that all the DL-based baselines missed.

Note that the numbers for DLFix are different from those
reported in DLFix paper~\cite{icse20}. The reason is that DLFix's
authors had filtered from the datasets the multiple-line bugs due to
DLFix's limited capabability. We used the full datasets with all the
bugs.

%Tien
%The reason for the difference between the results for DLFix in this paper and those reported in DLFix’s paper_[18] is that in DLFix’s experiment, the authors filtered from the datasets to keep only the single-line bugs for the comparison with other baselines. In CDFix, we used the full datasets with multiple-line bugs. Thus, the results reported in this paper are lower. The same reason is for other baselines.



%{\footnotesize{
\begin{table}[t]
  \caption{RQ1. Comparison Results with DL-based APR Approaches on Defects4J \underline {without Fault Localization}.}
  \vspace{-6pt}
  {\small
			\begin{center}
				\renewcommand{\arraystretch}{1}
				\begin{tabular}{p{0.9cm}<{\centering}|p{1.4cm}<{\centering}|p{1cm}<{\centering}|p{1cm}<{\centering}|p{1cm}<{\centering}|p{1cm}<{\centering}}
					
					\hline
					&\textbf{SequenceR}&\textbf{DLFix}& \textbf{Coconut}&\textbf{CURE}&\textbf{\tool}\\
					\hline
					Chart  & 4/5   & 7/13  & 8/13  & 7/12   & 9/12\\
					Closure& 5/7   & 7/12  & 7/18  & 9/27   & 12/24\\
					Lang   & 2/2   & 6/15  & 6/16  & 7/12   & 9/16\\
					Math    & 8/11  & 18/28 & 20/31 & 21/33  & 22/34\\
					Mockito & 0/0   & 1/1   & 2/3   & 2/3    & 2/4\\
					Time    & 0/0   & 1/3   & 2/4   & 3/5    & 3/6\\
					\hline
					Total   & 19/25 & 40/72 & 44/85 & 48/92  & 56/96\\
					\hline
					P(\%)  & 76.0  & 55.6  & 51.8  & 52.2   & 58.3\\
					\hline
				\end{tabular}
			{\footnotesize{
				Note: P is the probability of the generated plausible patches to be correct.\\
				In the cells, x/y: x means the number of correct fixes and y means the number of candidate patches that can pass all test cases. For example, for \tool, 96 candidate patches can pass all test cases. However, 56 out of 96 are the correct fixes compared with the fixes by developers in the ground truth.}}
				\label{RQ1_defect4j}
			\end{center}
                }
		\end{table}
%}}


\begin{table}[t]
  \caption{RQ1. Comparison Results with DL-based APR Approaches on Defects4J \underline {with Fault Localization}.}
  \vspace{-6pt}
  {\small
			\begin{center}
				\renewcommand{\arraystretch}{1}
				\begin{tabular}{p{0.9cm}<{\centering}|p{1.4cm}<{\centering}|p{1cm}<{\centering}|p{1cm}<{\centering}|p{1cm}<{\centering}|p{1cm}<{\centering}}
					
					\hline
					&\textbf{SequenceR}&\textbf{DLFix}& \textbf{Coconut}&\textbf{CURE}&\textbf{\tool}\\
					\hline
					Chart  & x/x   & x/xx  & x/xx  & x/xx   & x/xx\\
					Closure& x/x   & x/xx  & x/xx  & x/xx   & xx/xx\\
					Lang   & x/x   & x/xx  & x/xx  & x/xx   & x/xx\\
					Math    & x/xx  & xx/xx & xx/xx & xx/xx  & xx/xx\\
					Mockito & x/x   & x/x   & x/x   & x/x    & x/x\\
					Time    & x/x   & x/x   & x/x   & x/x    & x/x\\
					\hline
					Total   & xx/xx & xx/xx & xx/xx & xx/xx  & xx/xx\\
					\hline
					P(\%)  & xx.x  & xx.x  & xx.x  & xx.x   & xx.x\\
					\hline
				\end{tabular}
				\label{RQ1_defects4J_with_FL}
			\end{center}
                }
		\end{table}















%========================================end ===========================


\iffalse
{\footnotesize{
		\begin{table}[t]
			\caption{RQ1. Comparison with the Pattern-based APR Baselines on Defect4J.}
			\begin{center}
				\renewcommand{\arraystretch}{1}
				\begin{tabular}{p{0.8cm}<{\centering}|p{0.6cm}<{\centering}|p{1.1cm}<{\centering}|p{0.8cm}<{\centering}|p{1cm}<{\centering}|p{0.6cm}<{\centering}|p{0.8cm}<{\centering}}
					
					\hline
					&\textbf{Tbar}&\textbf{SequenceR}&\textbf{DLFix}& \textbf{Coconut}&\textbf{CURE}&\textbf{\tool}\\
					\hline
					Chart  & 11/13  & 4/5   & 7/13  & 8/13  & 7/12   & 9/12\\
					Closure& 17/26  & 5/7   & 7/12  & 7/18  & 9/27   & 12/24\\
					Lang   & 13/18  & 2/2   & 6/15  & 6/16  & 7/12   & 9/16\\
					Math   & 22/35  & 8/11  & 18/28 & 20/31 & 21/33  & 22/34\\
					Mockito& 3/3    & 0/0   & 1/1   & 2/3   & 2/3    & 2/4\\
					Time   & 3/6    & 0/0   & 1/3   & 2/4   & 3/5    & 3/6\\
					\hline
					Total  & 69/101 & 19/25 & 40/72 & 44/85 & 48/92  & 56/96\\
					\hline
					P(\%)  & 68.3   & 76.0  & 55.6  & 51.8  & 52.2   & 58.3\\
					\hline
				\end{tabular}
				Note: P is the probability of the generated plausible patches to be correct.\\
				In the cells, x/y: x means the number of correct fixes and y means the number of candidate patches that can pass all test cases. For example, for \tool, 96 candidate patches can pass all test cases. However, 56 out of 96 are the correct fixes compared with the fixes in the ground truth.
				\label{RQ1_defect4j}
			\end{center}
		\end{table}
}}
\fi
