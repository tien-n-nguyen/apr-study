\subsection{\bf RQ1. Comparison Results with DL-based APR Approaches on Defects4J Dataset}

\begin{table}[t]
  \caption{RQ1. Comparison Results with DL-based APR Approaches on Defects4J (395 bugs) \underline {without Fault Localization}.}
  \vspace{-6pt}
  \tabcolsep 2.5pt
  {\footnotesize
			\begin{center}
				\renewcommand{\arraystretch}{1}
				\begin{tabular}{p{0.7cm}<{\centering}|p{0.6cm}<{\centering}|p{0.6cm}<{\centering}|p{0.95cm}<{\centering}|p{0.7cm}<{\centering}|p{0.7cm}<{\centering}|p{0.9cm}<{\centering}|p{0.5cm}<{\centering}|p{0.7cm}<{\centering}|p{0.7cm}<{\centering}}
					
					\hline
					&\textbf{Seq.R}&\textbf{DLFix}& \textbf{CocoNuT}&\textbf{CURE}&\textbf{CURE*}&\textbf{Recoder}& {\bf ReR} & {\bf DEAR} & \textbf{\tool}\\
					\hline
					Chart  & 4/5   & 7/13  & 8/17  & 7/18  & 9/20 & 11/? & ? & 9/17 & 9/18\\
					Closure& 5/7   & 7/12  & 7/14  & 9/21 & 11/19  & 26/? & ? & 12/17 & 12/18\\
					Lang   & 2/2   & 6/15  & 6/16  & 7/12 & 8/18 & 10/? & ? & 8/16 & 9/16\\
					Math    & 8/11  & 18/28 & 20/31 & 21/33 & 21/35 & 19/? & ? & 20/33 & 22/34\\
					Mockito & 0/0   & 1/1   & 2/3   & 2/3  &2/4 & 2/? & ? & 2/4 & 2/4\\
					Time    & 0/0   & 1/3   & 2/4   & 3/5  &3/7 & 3/? & ? & 3/5 & 3/6\\
					\hline
					Total   & {\bf 19}/25 & {\bf 40}/72 & {\bf 44}/85 & {\bf 48}/92 & {\bf 54}/103 & {\bf 71/?} & {\bf 45}/? &{\bf 53}/91 & {\bf 56}/96\\
					\hline
					P(\%)  & 76.0  & 55.6  & 51.8  & 52.2  &  53.4 & ? & & 58.2 & 58.3\\
					\hline
				\end{tabular}
			{\footnotesize{
				X/Y: X is the number of correct fixes, Y is the number of candidate patches passing all test cases. P: probability of the correct generated plausible patches. CURE: 5-hour validation, CURE*: 5,000 candidates.?: not reported., ReR: RewardRepair}}

%                               For example, for \tool, 96 candidate patches can pass all test cases, but only 56 of them match with developers' fixes.}}
%                                However, 56 out of 96 are the correct fixes compared with the fixes by developers in the ground truth.}}
%			{\color{red}{CURE: the cure approach with 5 hours limitation per bug which is the same setting as our \tool. CURE*: the cure approach with their own setting that validating the top 5,000 candidate patches per bug}}
				\label{RQ1_defect4j}
			\end{center}
                }
		\end{table}

\subsubsection{{\bf Without Fault Localization}}

{\em {\tool} fixes more bugs than DL-based baselines} except
Recoder. {\tool} auto-fixes 56 bugs, i.e., 194.7\% (37 bugs), 40\%
(16), 27.3\% (12), 16.7\% (8), 3.7\% (2), 24.4\% (11), and 5.6\% (3) more bugs than
SequenceR, DLFix, CoCoNuT, CURE, CURE*, RewardRepair, and DEAR respectively
(Table~\ref{RQ1_defect4j}).

Compared to Recoder, {\tool} complements to Recoder. Since generating
only {\em single-hunk} patches, Recoder did not fix 13
multi-hunk/multi-statement bugs that {\tool} fixed (Table~\ref{multi:tab}).
{\tool} can handle the multi-hunk/multi-statement bugs
%because the hunk
%detection and statement expansion were used as in DEAR.
thanks to its ability to learn the transformations to multiple buggy
AST subtrees in the same context.
%
Recoder fixed 28 single-hunk bugs that {\tool} missed. Recoder did not
report the number of plausible patches.

%With the 5-hour limit, {\tool} generates {\em the most correct (56)
%  and the most plausible patches (96)} than other baselines.  {\tool}
%also has a {\em higher percentage of the generated plausible patches
%  to be correct than all the baselines}, except SequenceR.  However,
%{\tool} can fix 194\% more bugs and 6 times more plausible patches
%than SequenceR.  {\tool} also fixed 3 bugs that all the DL-based
%baselines missed.

Compared to CURE, with the same setting of 5-hour limit, {\tool} fixes
16.7\% (8 bugs) more than CURE. {\tool} fixed 12 bugs that
CURE missed; and CURE fixed only 4 bugs that {\tool} missed.
%
CURE*, with the cut-off setting, ranked 10,000 candidates and refined
the ranking to pick and validate top 5,000 candidates. Despite taking
more patch validation time (5.5 hours on average for
validating those 5,000 candidates), CURE* fixed 2 bugs less than
{\tool}.

%In comparison with CURE and CURE*, CURE* ranked 10,000 candidates and
%refined the ranking to pick and validate top 5,000 candidates. Despite
%taking more time of patch validation (about 5.5 hours), CURE* fixed 2
%bugs less than {\tool}. However, with the same setting of 5-hour
%limit, {\tool} fixes 16.7\% (8 bugs) more than CURE. Moreover, {\tool}
%fixed 12 bugs that CURE missed; and CURE fixed only 4 bugs that
%{\tool}~missed.

%Despite costing more than double time (>10 hours of validation), CURE*
%fixed 2 bugs less than {\tool}. With 5-hour limit, if CURE did not
%identify a plausible patch, on average, it validated 2,350
%candidates. Importantly, {\tool} fixes 16.7\% (8 bugs) more than CURE
%with the same 5-hour limit. Moreover, {\tool} fixed 12 bugs that CURE
%missed; and CURE fixed 4 bugs that {\tool}~missed.

%NEW
\begin{table}
	\caption{RQ1. Detailed Comparison with DL-based APR Models on Defects4J \underline{without} Fault Localization}
	\vspace{-7pt}
	\begin{center}
	  \small
          \tabcolsep 2.5pt
		\renewcommand{\arraystretch}{1} 
%		\begin{tabular}{p{4cm}<{\centering}|p{1cm}<{\centering}|p{1cm}<{\centering}|p{1cm}<{\centering}}
\begin{tabular}{p{2.9cm}<{\centering}|p{0.7cm}<{\centering}|p{1.1cm}<{\centering}|p{0.6cm}<{\centering}|p{0.8cm}<{\centering}|p{0.7cm}<{\centering}|p{0.7cm}}
			\hline
			Bug Types & DLFix& CoCoNuT & CURE & Recoder & DEAR & {\tool}\\\hline
			
			One-Hunk One-Stmt  & 35 & 37 & 38 & 64 & 33 & 36\\
			One-Hunk Multi-Stmts  & 1 &3 & 3 & 4 & 4 & 4\\ 
			Multi-Hunks One-Stmt  & 4 &3 & 6 & 3 & 13 & 13 \\
			Multi-Hunks Multi-Stmts  & 0 &0 & 0 & 0 & 1 & 1\\
			Multi-Hunks Mix-Stmts  & 0 & 1 & 1 & 0 & 2 & 2\\\hline
			Total & 40 & 44 & 48 & 71 & 53 & {\bf 56}\\
			\hline
		\end{tabular}
		%		Type 1: A bug with the fix involving only one hunk and one statement in the hunk.
		%		Type 2: A bug with the fix involving only one hunk and multiple statements in that hunk.
		%		Type 3: A bug with the fix involving multiple hunks and each hunk having one fixed statement.
		%		Type 4: A bug with the fix involving multiple hunks and each hunk having multiple statements.
		%		Type 5: A bug with the fix involving multiple hunks and some hunks having one statement and some others having multiple ones.
		%		$GenFix_{T}$: Regarding each hunk has one buggy statement and the use test case to verify. If no solution found, we do the expansion and fix again.
		\label{multi:tab}
	\end{center}
	\vspace{-4pt} %-3pt
\end{table}



%\vspace{-6pt}
\subsubsection{{\bf With Fault Localization}}

Table~\ref{RQ1_defects4J_with_FL} shows the result on
Defects4J with an FL tool. Due to the confounding inaccuracy of the FL
tool, all models decrease accuracies. {\tool} fixed more bugs
than any baseline except Recoder. It automatically fixes 44 bugs,
i.e., 193.3\% (29 bugs), 46.7\% (14), 33.3\% (11), 22.2\% (8), and
10\% (4) more bugs than the baselines SequenceR, DLFix, CoCoNuT, CURE,
and CURE*, respectively.

Compared to Recoder, {\tool} complements to Recoder.  Because
generating only {\em single-hunk} patches, Recoder did not fix 9
multi-hunk/multi-statement bugs that {\tool} fixed
(Table~\ref{multi-tab-with-FL}). Recoder fixed 14 single-hunk bugs that {\tool} missed.

%With 5-hour limit, {\tool} fixed the most correct (44) and the most
%plausible (77) patches. Moreover,
Compared to CURE, with 5-hour limit, {\tool} fixed 9 bugs that CURE
missed; CURE fixed 3 bugs that {\tool} missed. {\tool} fixed
2 bugs that all the baselines missed. For CURE* with the
cut-off setting, {\tool} takes 10\% less time (0.5 hours less on
average), but fixed 10 more bugs than CURE*.


%^CURE* costs more than double validation
%time, but fixed 4 bugs less than {\tool}.

%Furthermore, {\tool} generates the most plausible patches (i.e., {\bf
%  XX}) among all the models.  It also has a higher percentage of the
%generated plausible patches to be correct than all the
%baselines. Moreover, CDFix fixed {\bf 9} bugs that the {\em best
%  baseline} CURE missed; and CURE fixed {\bf 3} bugs that CDFix
%missed. CDFix fixed {\bf 2} bugs that all the DL-based baselines
%missed.

%
%Note that the numbers for DLFix are different from those
%reported in DLFix paper~\cite{icse20}. The reason is that DLFix's
%authors had filtered from the datasets the multiple-line bugs due to
%DLFix's limited capabability. We used the full datasets with all the
%bugs.

%Tien
%The reason for the difference between the results for DLFix in this paper and those reported in DLFix’s paper_[18] is that in DLFix’s experiment, the authors filtered from the datasets to keep only the single-line bugs for the comparison with other baselines. In CDFix, we used the full datasets with multiple-line bugs. Thus, the results reported in this paper are lower. The same reason is for other baselines.




%}}




\begin{table}[t]
  \caption{RQ1. Comparison Results with DL-based APR Approaches on Defects4J (395 bugs) \underline {with Fault Localization}.}
  \vspace{-6pt}
         {\footnotesize
        \tabcolsep 2.5pt
			\begin{center}
				\renewcommand{\arraystretch}{1}
				\begin{tabular}{p{0.7cm}<{\centering}|p{0.6cm}<{\centering}|p{0.6cm}<{\centering}|p{1cm}<{\centering}|p{0.7cm}<{\centering}|p{0.7cm}<{\centering}|p{0.8cm}<{\centering}|p{0.5cm}<{\centering}|p{0.7cm}<{\centering}|p{0.7cm}<{\centering}}
					
					\hline
					&\textbf{Seq.R}&\textbf{DLFix}& \textbf{CocoNuT}&\textbf{CURE}&\textbf{CURE*}& {\bf Recoder} & {\bf ReR} & {\bf DEAR} &\textbf{\tool}\\
					\hline
					Chart  & 3/3   & 5/12  & 6/11  & 6/13  & 6/15 & 8/14& ? & 8/16 & 9/16\\
					Closure& 4/5   & 6/10  & 6/9   & 6/10  & 7/14 & 17/31 & ? & 7/11 & 9/15\\
					Lang   & 2/2   & 5/12  & 5/13  & 5/14  & 6/14  & 9/15 & ?  & 8/15 & 7/13\\
					Math    & 6/9  & 12/18 & 13/21 & 16/23 & 17/30 & 15/30 & ? & 20/33 & 21/27\\
					Mockito & 0/0   & 1/1   & 2/2   & 2/2  & 2/3  & 2/2 & ? & 1/2 & 1/3\\
					Time    & 0/0   & 1/2   & 1/1   & 1/2  & 2/3 & 2/2 &? & 3/6 & 3/3\\
					\hline
					Total   & {\bf 15}/19 & {\bf 30}/55 & {\bf 33}/57 & {\bf 36}/71 & {\bf 40}/79 & {\bf 53}/94 & {\bf 29}/? & {\bf 47}/91 & {\bf 50}/77\\
					\hline
					P(\%)  & 68.4  & 54.5  & 57.9  & 50.7  & 50.6 & 56.4& ? & 51.2 & 57.1\\
					\hline
				\end{tabular}
				\label{RQ1_defects4J_with_FL}
			\end{center}
                }
		\end{table}


\begin{table}
	\caption{RQ1. Detailed Comparison with DL-based APR Models on Defects4J \underline{with} Fault Localization}
	\vspace{-10pt}
	\begin{center}
        \small
          \tabcolsep 2.5pt
		\renewcommand{\arraystretch}{1} 
%		\begin{tabular}{p{4cm}<{\centering}|p{1cm}<{\centering}|p{1cm}<{\centering}|p{1cm}<{\centering}}
\begin{tabular}{p{2.9cm}<{\centering}|p{0.7cm}<{\centering}|p{1.1cm}<{\centering}|p{0.6cm}<{\centering}|p{0.8cm}<{\centering}|p{0.7cm}<{\centering}|p{0.7cm}}
			\hline
			Bug Types & DLFix& CoCoNuT & CURE & Recoder & DEAR & {\tool}\\\hline
			
			One-Hunk One-Stmt  & 30 & 33 & 36 & 44 & 29 & 32\\
			One-Hunk Multi-Stmts  & 0 &0 & 0 & 6 & 4 & 4\\ 
			Multi-Hunks One-Stmt  & 0 &0 & 0 & 3 & 11 &  11\\
			Multi-Hunks Multi-Stmts  & 0 &0 & 0 & 0 & 1 & 1\\
			Multi-Hunks Mix-Stmts  & 0 & 0 & 0 & 0 & 2 & 2\\\hline
			Total & 30 & 33 & 36 & 53 & 47 & {\bf 50}\\
			\hline
		\end{tabular}
		\label{multi-tab-with-FL}
	\end{center}
\vspace{-3pt}
\end{table}












%========================================end ===========================




