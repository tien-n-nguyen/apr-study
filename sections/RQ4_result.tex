\subsection{\bf RQ4. Impact of Dual-Task Learning}


\begin{table}[t]
  \caption{RQ4.Impact Analysis Results of Dual-Task Learning on Performance (running on BigFix Dataset).}
  \vspace{-6pt}
	{\small
	  \begin{center}
            \tabcolsep 3pt
			\renewcommand{\arraystretch}{1}
			\begin{tabular}{p{1cm}<{\centering}|p{3.2cm}<{\centering}|p{2cm}<{\centering}|p{1cm}<{\centering}}
				\hline
				Top-$K$ & Transformation-only Model & Cascading Model &  \tool \\			
				\hline
				Top-1   & 7.1\% & 11.7\% & 14.9\% \\ \hline
				Top-5	& 8.9\% & 13.1\% & 16.1\% \\ \hline
				Top-10	& 9.7\% & 14.3\% & 16.8\%\\ \hline
			
				\hline
			\end{tabular}
			\label{fig:rq4_results}
		\end{center}
	}
\end{table}


Table~\ref{fig:rq4_results} shows the comparison among {\tool} and its
variants. As seen, the top-$K$ values of the
\code{Transformation-only} model are 52.3\%, 44.7\% and 42.3\% lower
than those of {\tool} in Top-1, Top-5, and Top-10 values,
respectively. This result shows that {\em context learning}
contributes positively in {\tool}'s performance, and our {\em dual-task
learning helps propagate the impact from CCL (context learning) to CTL
(transformation learning)} to improve {\tool}.


%result on the impact of our dual learning architecture on the overall
%{\tool}'s bug-fixing performance.  As seen, the top-$K$ values of the
%\code{Transformation-only} model are 52.3\%, 44.7\% and 42.3\% lower
%than those of {\tool} in Top-1, Top-5, and Top-10, respectively. This
%result shows that 1) the context learning model in {\tool} has good
%impact on the overall performance, and 2) the dual learning enables
%the impact from context learning to transformation learning to achieve
%high performance in APR.

As seen in Table~\ref{fig:rq4_results}, the top-$K$ values of the
\code{Cascading} model are 21.5\%, 18.6\%, and 14.9\% lower than those
of {\tool} in Top-1, Top-5, and Top-10, respectively. This result
shows that the {\em cascading architecture between context learning (CCL)
and transformation learning (CTL) is not as effective as 
dual-task learning in {\tool}}.

We also performed overlapping analysis between the results from
{\tool} and the \code{Cascading} model. {\tool} fixes {\bf 54} bugs
that the cascading model missed and the \code{Cascading} model fixed
only {\bf 17} bugs that {\tool} missed, while both models fix the same
{\bf 119} bugs. This result shows that {\tool} can fix more than 3
times unique bugs that the \code{Cascading} model missed than the
unique bugs that were fixed by the \code{Cascading} model but missed
by {\tool}.

%other way.



In another study, we analyzed the set T1 of {\bf 418} bugs in which
the {\em outputs of the context learning model in the \code{Cascading}
  model are incorrect} compared to the oracle. We also analyzed the
set T2 of the {\bf 789} bugs in which {\em the outputs of the context
  learning model in {\tool} with dual-task learning are correct}
compared to the oracle. A correct match is defined as $\geq$ 90\%
matching of all AST nodes in the context, otherwise, it is an
incorrect match. Among the overlapping bugs in T1 $\cap$ T2 (i.e., the
bugs that {\tool} correctly learns the context while the
\code{Cascading} model did not), we reported {\bf 134} bugs that
     {\tool} was able to fix, and {\bf 89} bugs that the
     \code{Cascading} model fixed.  This result indicates that {\bf
       the correct context learning (thanks to {\tool}'s dual-task
       learning) leads to more correct bug fixing}.

We further analyzed {\bf 54} bugs that {\tool} fixed and were missed
by the \code{Cascading} model. Among them, we found {\bf 46} bugs in
which {\em the outputs of context learning (CCL) in {\tool} match with the fixed contexts
  in the oracle}. In contrast, we found {\bf 18} bugs in which {\em
  the outputs of context learning (CCL) in the \code{Cascading} model match with the
  fixed contexts}. This indicates that one source of the inaccuracy in
the \code{Cascading} model is the inaccuracy of context learning. This
also shows that {\bf the improvement of {\tool} over the
  \code{Cascading} model comes from dual-task learning, which makes
  the context learning more correct, leading to more correct
  bug-fixing}.

%improvement in its fixing capability for the cases that the cascading
%model missed comes from the dual-task learning, which makes the
%context learning more correct, leading to more correct bug-fixing.

%In other words, dual-task learning helps the propagation of the mutual
%impact between context learning and transformation learning, leading
%to the improvement in APR.

{\em Relation between DLFix~\cite{icse20} and the \code{Cascading}
  model.} The \code{Cascading} model differs from
DLFix~\cite{icse20}. First, CCL (context) and CTL (transformation) are
different from those in DLFix, which uses code summarization. Second,
in the \code{Cascading} model, the output of CCL corresponding to a
buggy subtree is directly used as the input of CTL. In
DLFix~\cite{icse20}, the summarized vector is used as a weight in a
cross-product to represent the impact from CCL to CTL.

Although DLFix differs from the \code{Cascading} model in the CCL and
CTL components as well as in the ways that they connect, they share
the same principle of cascading architecture. Thus, the above result
could serve as an explanation on the reason of {\tool} improving over
DLFix~\cite{icse20}: the dual-task learning makes the context learning
more accurate, and as a result, more correct bug-fixing.

%the correctness of the context learning model.



%Tien
%This result shows that 1) the cascading architecture between context
%learning (CCL) and transformation learning (CTL) is not effective as
%the dual-learning architecture as in {\tool}, and 2) dual learning
%between CCL and CTL is effective and helps improve APR
%performance. This result also explains the reason for the higher
%performance of {\tool} over the state-of-the-art APR approach in
%DLFix~\cite{icse20}, which has a cascading architecture of context
%learning and transformation learning.


%Table~\ref{fig:rq4_results} presents the results of contributions of dual-learning in CDFix. The results show that Only-transformation-model reduces 52.3\%, 44.7\% and 42.3\% of {\tool} using Top-1, Top-5, and Top-10, respectively, which indicates that context-learning model is important to our {\tool}.

%The Cascading model also reduces the Top-1, Top-5, and Top-10 of {\tool} by 21.5\%, 18.6\%, and 14.9\%, respectively, indicating that the simultaneous dual-learning of context learning model and transformation learning model is effective.
