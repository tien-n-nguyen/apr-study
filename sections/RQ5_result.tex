\subsection{\bf RQ5. Evaluation on C/C++ Projects.}




\begin{table}[t]
	\caption{RQ4. Codeflaws (C/C++ Projects) versus Defects4J (Java Projects). P\% = $|$Fixed Bugs$|$/\{Plausible Patches\}.}
	\vspace{-5pt}
	{\small
		\begin{center}
			%\renewcommand{\arraystretch}{1}
			\tabcolsep 2.7pt
			\begin{tabular}{p{0.3cm}<{\centering}|p{1.5cm}<{\centering}|p{1.55cm}<{\centering}|p{0.3cm}<{\centering}|p{1.5cm}<{\centering}|p{1.55cm}<{\centering}}\hline	
				%	Metric& BugsInPy (Python projects) &  Defects4J (Java projects)\\\hline
				
				\multicolumn{3}{c|}{Codeflaws } & \multicolumn{3}{c}{Defects4J}\\\hline
				 P\%& \# of Fixed Bugs& \# of Plausible Patches &P\%& \# of Fixed Bugs & \# of Plausible Patches \\ \hline
				
				   &  &&&&\\ 
				\hline

				%\vspace{1pt}
			\end{tabular}
			\label{RQ5}
		\end{center}
	}
	\vspace{-10pt}
\end{table}


As seen in Table~\ref{RQ5}, {\tool} can autofix xx bugs in C/C++ projects of Codeflaws. 
The empirical results show that the performance of
{\tool} on the C/C++ projects is consistent with the one on the Java
projects. Specifically, the percentages of the C/C++ and Java bugs that can be fixed are similar, e.g., xx\% vs. xx\%, respectively. 