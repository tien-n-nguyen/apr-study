\subsection{\bf RQ5. Evaluation on C/C++ Projects}
\label{sec:eval-c}

\begin{table}[t]
	\caption{RQ4. Codeflaws (C/C++) versus Defects4J (Java). P\% = $|$Fixed Bugs$|$/\{Plausible Patches\}.}
	\vspace{-5pt}
	{\footnotesize
		\begin{center}
			%\renewcommand{\arraystretch}{1}
			\tabcolsep 2.7pt
			\begin{tabular}{p{0.5cm}<{\centering}|p{1.5cm}<{\centering}|p{1.55cm}<{\centering}|p{0.5cm}<{\centering}|p{1.5cm}<{\centering}|p{1.55cm}<{\centering}}\hline	
				%	Metric& BugsInPy (Python projects) &  Defects4J (Java projects)\\\hline
				
				\multicolumn{3}{c|}{Codeflaws (3,902 bugs)} & \multicolumn{3}{c}{Defects4J (395 bugs)}\\\hline
				 P\%& \# of Fixed Bugs& \# of Plausible Patches &P\%& \# of Fixed Bugs & \# of Plausible Patches \\ \hline
				
				  76.3  &        513 (13.1\%)      &            672             &  58.3 &          56 (14.2\%)       &        96                 \\
				\hline

				%\vspace{1pt}
			\end{tabular}
			\label{RQ5}
		\end{center}
	}
%	\vspace{-10pt}
\end{table}
  
To evaluate {\tool} on the projects with a different programming
language, we ran it on Codeflaws, a dataset of bugs in C/C++ projects.
As seen in Table~\ref{RQ5}, {\tool} is able to correctly fix 513 bugs
in the total of 3,902 bugs in that dataset, with 672 plausible
patches. The percentage of the number of fixed bugs over the plausible
patches for C/C++ projects is higher than that for Java projects
(76.3\% versus 58.3\%). However, the percentage of the fixed bugs in
the total bugs in the each dataset is similar (13.1\% in Codeflaws
versus 14.2\% in Defects4J). In brief, the performance of {\tool} for
C/C++ projects in Codeflaws is consistent with that for Java projects
in Defects4J.

%This result also shows that {\tool} is able to work on
%any programming language with similar concepts used in {\tool} such as
%code tokens, AST, subtrees, tree transformations, and dual-task learning.

%are language-independent.



%As seen in Table~\ref{RQ5}, {\tool} can autofix xx bugs in C/C++
%projects of Codeflaws.  The empirical results show that the
%performance of {\tool} on the C/C++ projects is consistent with the
%one on the Java projects. Specifically, the percentages of the C/C++
%and Java bugs that can be fixed are similar, e.g., 13.1\% vs. 14.2\%,
%respectively.



