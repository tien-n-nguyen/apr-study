\subsection{Threats to Validity}



Our empirical results have the following threats to the validity. (1)
We evaluated our {\tool} on Java and C/C++ code.  The key modules in
{\tool} are language-independent and could potentially work for other programming
languages. (2) We re-implemented CURE as its code was not publicly
available online. Despite our effort to contact the authors of CURE,
we have never received their response. Under that, we tried our best
to follow the details in their paper during the implementation.
%(3) The result of our implemented CURE in this paper is lower than
%the one in its original paper (48 vs 57) on the same Defects4J
%dataset. The reasons can be two folds. First, in the validation step
%of the original CURE, the authors first ranked 10,000 candidates and
%refined the ranking to pick top 5,000 candidates, then validated each
%candidate.
(3) For CURE, we have two settings for patch validation: 5-hour limit
and 5,000 cutoff on the number of candidates. For some other DL-based
baselines, we do not have the code, thus, cannot run them with the
5,000 candidate cutoff setting. Thus, we chose 5-hour limit setting is
the main one, while still running CURE in both settings for
comparison.
%In our setting, we set the 5-hour time limit on the validation step,
%instead of a cut-off on the number of candidates. The rationale for
%our setting is that it is the one used in all other studies of the
%baseline models. In the cases that our implemented CURE could not
%identify a plausible patch after 5-hour limit, on average, we only
%looped through about ~2,350 candidates, which is far fewer than
%5,000. Second, we used BigFix as our training dataset, which is
%different from the one used in the original paper of CURE. In brief,
%for all DL-based approaches, we used the same datasets and under the
%same settings and computing environments.

Note that the numbers for DLFix are different from those reported in
DLFix paper~\cite{icse20}. The reason is that DLFix's authors had
filtered from the datasets the multiple-line bugs due to DLFix's
limited capabability. We used the full datasets with all the bugs.



%The compared baselines and FixLocator require a dataset with test cases for generating code coverage information, run-time information, and failing traces, thus we only compared them on Defects4J. In the future, we plan to test FixLocator on more types of datasets. 
%(3) The DeepFL is designed for method-level fault localization. To compare it with FixLocator fairly, we use only part of its features that are applicable to statement level fault localization. The other baselines CNNFL and DeepRL4FL can be directly applied on statement-level FL. 
%(4) To compare fairly, we evaluated the baselines using out proposed Hit-N@Set metrics. However, we also tested our FixLocator using the ranking metrics: Hit-N@Top-K. FixLocator is consistently better than all baselines in any metric.
