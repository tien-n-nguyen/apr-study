\section{Introduction}

%Fixing software defects is one of the crucial maintenance
%activities. Thus,

Detecting and fixing software defects is one of the most crucial
activities in software development. Several researchers have
developed the approaches to automate that process. The approaches that
help developers automatically fix a software defect in a program is
referred to as {\em automated program repair} (APR). The APR
approaches can be broadly classified in the following categories
according to their techniques:
%Researchers have proposed several approaches to help developers in
%automatically identifying and fixing the defects in programs. Such
%approaches are referred to as {\em automated program~repair}
%(APR). The APR approaches have been leveraging various techniques in
%the areas of
{\em search-based software engineering}, {\em software mining}, {\em
  machine learning (ML)}, and {\em deep learning (DL)}.

In search-based
  approaches~\cite{LeGoues-icse12,le2011genprog,martinez2016astor,qi2014strength},
the buggy code is first mutated via certain operators to produce the
potential solution space. Then, a search strategy is designed to find
the fix in the solution space. Instead searching for a solution, the
software mining-based APR models aim to learn the fixing patterns from
the prior bug
fixes~\cite{kim2013automatic,le2016history,liu2019avatar,tbar-issta19,nguyen2013semfix,
  icse10,ray-fse12}. Some mining-based approaches learn the patterns
from the source code~\cite{liu2019avatar,tbar-issta19} and others
learn them from prior bug-fixing 
changes~\cite{wen2018context,Simfix,koyuncu2018fixminer}.  The fixing
patterns/templates could be mined automatically from the repositories
or pre-defined via semi-automated
techniques~\cite{le2016history,nguyen2013semfix,liu2019avatar,tbar-issta19}.

Recent advances in machine learning (ML) enable several models to
implicitly learn from prior bug fixes to apply to repair the current
buggy code. Then, they derive the candidate fixes and rank them
according to their
likelihoods~\cite{long2016automatic,long2017automatic,saha2017elixir}.
With the revolutionary advances in deep learning (DL), software
engineering researchers have leverage them to automatically derive the
fixing changes to a given buggy code. Some DL-based approaches learn
the fix from prior {\em similar bug fixes and/or
  patterns}~\cite{gupta2017deepfix,white2019sorting,white2016deep}.
Other DL-based approaches treat APR as {\em machine translation} from
a buggy version to a correct one with some DL models including
transformers and
others~\cite{chakrabortycodit,chen2018sequencer,hata2018learning,tufano2018empirical,see2017get}. Instead
of treating APR as machine translation, other DL-based approaches aim
to implicitly learn the {\em transformation rules} from a buggy code
to a correct one accordingly to the surrounding {\em context} of the
transformations~\cite{icse20,tufano2019learning,cure-icse21}. Those
recent DL-based APR approaches have showed that {\em learning
  bug-fixing changes needs to be dependent on the surrounding {\bf
    code context}}.  For example, a null check is needed after a call
to read data from a socket with \code{BufferedReader.readline} to make
sure a successful data retrieval. Those context-aware APR models have
achieved better performance in
automated program repair~\cite{icse20,tufano2019learning,chakrabortycodit}.

Despite recognizing the importance of contexts in learning the fixes,
the existing DL-based APR approaches still have limitations in
integrating contextual information in the APR
process. \underline{First}, the DL-based APR approaches that learn
from {\em bug-fixing patterns}~\cite{white2016deep,gupta2017deepfix}
have focused on similar code and/or code changes with {\em little or
  no consideration} on whether those code or fixing patterns appear in
certain surrounding contexts. \underline{Second}, the DL-based APR
approaches that leverage {\em machine translation or
  transformers}~\cite{chakrabortycodit,chen2018sequencer,hata2018learning,tufano2018empirical,see2017get}
often take too little surrounding code as context to learn fixing
changes or do not have a clear boundary of the fixing changes and the
context. Let us elaborate this point. Some DL-based APR approaches aim
to learn {\em only the code changes} for a fix, e.g., from one buggy
subtree in an Abstract Syntax Tree (AST) or a buggy statement to the
correct subtree or statement~\cite{chakrabortycodit}. In this
treatment, too little context might not help a model learn the fixes
that depend on a larger surrounding code. In contrast, other
approaches~\cite{chen2018sequencer,hata2018learning} take the entire
buggy method and translate it to the correct one, while the fix might
be only a small editing change to a single statement. Those
translation-based approaches {\em do not distinguish clearly the
  boundary} of the fixing change (e.g., to a buggy statement) and the
surrounding context (e.g., the preceeding or succeeding code). Due to
the mixture of fixing changes and context, those machine translation
models or transformers might not learn what fixing changes are
appropriate in specific contexts~\cite{icse20}.

\underline{Third}, to address that issue, DLFix~\cite{icse20} makes a
clear boundary of fixing changes and the surrounding context, and
dedicates two layers for those two tasks. The first layer is a
tree-based RNN model that learns the contexts of bug fixes and its
result is used as an additional weighting input for the second layer
designed to learn the bug-fixing code transformations. However, the
cascading architecture in DLFix from the two layers create a
comfounding effect from the inaccuracy of the learning of the context
to the learning of the bug-fixing code transformations.

In this work, we conjecture that the two tasks of learning the code
context and learning the bug-fixing code transformations are related
and dependent on each other. {\bf Correct learning of contexts can
  benefit the learning of code transformations and vice versa in
  auto-fixing context-dependent bugs}. For example, in C code, if the
preceding code (i.e., part of the context) contains \code{fopen} as in
\code{FILE *fd = fopen (fname, ``rb'');}, then the succeeding
bug-fixing code is more likely to be \code{if (fd != null)} or
\code{if (fd == null)}. However, if the preceding code contains
\code{fread} as in \code{n = fread(buffer,1,size,fd);}, then the
succeeding bug-fix is more likely to be \code{if (n != size)} or
\code{if (n == size)}, rather than a null check (\code{if (n !=
  null)}). In contrast, if the bug fix is a
change from \code{if (fd == null)} into \code{if (fd != null)}, then the
preceding code more likely contains \code{fopen} than \code{fread}.
Let us call such relation between two tasks as {\em duality}.

We introduce {\tool}, a context-aware dual-task learning APR model,
which dedicates one model for learning the bug-fixing code
transformations (CTL) and another one for learning the corresponding
surrounding code contexts (CCL). Instead of cascading them, we train
the models simultaneously with softsharing their parameters to exploit
this duality. We apply a probabilistic correlation as a regularization
term in the loss function in the join training. ...

{\bf Technical overview} paragraph

{\bf Empirical Results overview} paragraph

The contributions of this paper are listed as follows:

%{\bf A. DL for APR:} {\tool} is the first DL APR that generates
%comparable and complementary results with powerful pattern-based
%tools, as recently published DL-based APR can only fix very few bugs
%on Defects4J. {\tool} helps confirm that further research on building
%advanced DL to improve APR is promising and valuable.

{\bf A. A Novel Dual-Task Learning APR Model:} A DL-based,
context-aware APR approach with a novel dual-task learning model that
exploits the duality of learning bug-fixing code transformations and
learning code contexts.

{\bf B. Empirical Results:} (Code and data are published~\cite{AutoFix2019})

1) {\bf Improving over recent Pattern-based APR}.  We
show {\tool} can auto-fix more bugs than the state-of-the-art
pattern-based APR models. Importantly, {\tool} is data-driven and do
not require explicitly learning of bug-fixing patterns.

%pattern-based APR tools, and its result is comparable and
%complementary to the ones from the two best pattern-based
%tools.

2) {\bf Improving over all the DL-based APR}. We evaluated {\tool}
against the most recent DL-based models to show our model's better
performance.
%{\tool} is able to detect 2.5 times more bugs than the best performing
%baseline.  {\tool} can fix 253 new bugs (out of 1158 in Bugs.jar) than
%all the other DL-based APR techniques combined.

%-------------------------------------------------------------

%{\bf 1. {\tool}: Novel DL-based fault localization approach} that
%derives the co-change fixing locations for a bug. Our idea is
%to treat such problem as a dual learning task with the joint training
%of the method-level and statement-level co-fixing learning models.

%{\bf 2. Novel graph-based representation learning with co-change
%  statements.} Our graph-based representation learning with GCN
%and the novel type of features in co-change statements enables
%the dual-task models learn derive co-change fixing locations.

%{\bf 3. Extensive empirical evaluation.} We evaluated {\tool} against
%the most recent FL models to show our model's better performance. Our
%replication package is available at~\cite{FixLocator2022}.
