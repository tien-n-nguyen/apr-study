\section{Introduction}

%Fixing software defects is one of the crucial maintenance
%activities. Thus,

%Detecting and fixing software defects is one of the most crucial
%activities in software development.
Researchers have developed the approaches to automate the
bug-fixing task. The approaches that help
developers automatically fix a software defect is
referred to as {\em automated program repair} (APR). The APR
approaches can be broadly classified in the following categories
based on their techniques:
%Researchers have proposed several approaches to help developers in
%automatically identifying and fixing the defects in programs. Such
%approaches are referred to as {\em automated program~repair}
%(APR). The APR approaches have been leveraging various techniques in
%the areas of
{\em search-based software engineering}, {\em software mining}, {\em
  machine learning (ML)}, and {\em deep learning (DL)}.

In search-based
approaches~\cite{LeGoues-icse12,le2011genprog,martinez2016astor,qi2014strength},
the buggy code is first mutated via certain operators to produce the
potential~solution space. A search strategy is designed to find
the fix in the solution space. Instead searching for a solution, the
software mining-based APR models learn the fixing patterns from
the prior bug
fixes~\cite{kim2013automatic,le2016history,liu2019avatar,tbar-issta19,nguyen2013semfix,
  icse10,ray-fse12}. Some mining-based approaches learn the patterns
from the source code~\cite{liu2019avatar,tbar-issta19} and others
learn them from prior bug-fixing
changes~\cite{wen2018context,Simfix,koyuncu2018fixminer}.  The fixing
patterns could be mined automatically from the repositories
or pre-defined via semi-automatic
techniques~\cite{le2016history,nguyen2013semfix,liu2019avatar,tbar-issta19}.

%Recent advances in
Machine learning (ML)
%enable several models to
has facilitated implicit learning from prior bug fixes to apply to repair
the current buggy
code~\cite{long2016automatic,long2017automatic,saha2017elixir}.
%They derive the candidate fixes and rank them according to their
%likelihoods
%researchers have leveraged deep learning (DL) to automatically derive
%the fixing changes to a given buggy code. Some
Some Deep Learning (DL)-based approaches learn the fix from prior {\em
  similar bug fixes and/or
  patterns}~\cite{gupta2017deepfix,white2019sorting,white2016deep}.
Other DL-based models treat APR as {\em machine translation} from a
buggy code to a correct one using transformers and other
models~\cite{chakrabortycodit,chen2018sequencer,hata2018learning,tufano2018empirical,see2017get}.
%Instead of treating APR as machine translation,
In addition, other DL-based approaches implicitly learn the {\em
  transformation rules} from a buggy code to a correct one accordingly
to the surrounding {\em context} of the
transformations~\cite{chen2018sequencer,icse20,cure-icse21,lutellier2020coconut}.
Those approaches show that {\em learning bug-fixing changes might be
  dependent on {\em code context}}, and they 
%For example, in C~code, if the preceding context contains \code{fopen}
%as in \code{FILE *fd = fopen (fname, ``rb'');}, the succeeding
%bug-fixing code is more likely to be \code{if (fd != null)} or
%\code{if (fd == null)}. However, if the preceding context contains
%\code{fread} as in \code{n} \code{=} \code{fread(buffer,1,size,fd);},
%the succeeding bug-fix is more likely to be \code{if (n} \code{!=}
%\code{size)} or \code{if (n} \code{==} \code{size)}, rather than a
%null check \code{if (n} \code{!=} \code{null)}.
%Those context-aware models
have achieved better APR
performance~\cite{icse20,lutellier2020coconut,cure-icse21}.

%For example, a {\em null check} is needed after a call to read data
%from a socket with \code{BufferedReader.readline} to make sure a
%successful data retrieval.
%tufano2019learning,chakrabortycodit}.

Despite recognizing the importance of contexts in learning the fixes,
the existing DL-based APR approaches over the time still have
limitations in {\em integrating context learning into code-change
  learning} in the APR process, leading to their ineffectiveness in
fixing context-dependent bugs.

First, the earlier DL-based approaches that learn from {\em bug-fixing
  patterns}~\cite{white2016deep,gupta2017deepfix} have focused on
similar code and/or code changes with {\em little or no consideration}
on whether those code or fixing patterns appear in certain surrounding
contexts. Second, in the same vein of taking too little or no context,
other DL-based APR approaches aim to learn {\em only the code changes}
for fixes, e.g., from a buggy subtree in an Abstract Syntax Tree (AST)
to the correct subtree \cite{chakrabortycodit,see2017get}. Despite
learning the code transformations for bug-fixing, taking no context
does not help a model learn the fixes that depend on a larger
surrounding code. In the third type of direction, some
approaches~\cite{hata2018learning,tufano2019learning,tufano2018empirical}
take too large context. They leveraged machine translation to take the
entire buggy method and translate it to the correct one, while the fix
might be only a small editing change to a single statement. Those
translation-based approaches do not distinguish clearly the boundary
of the fixing change (e.g., to a buggy statement) and the surrounding
context. Due to the mixture of fixing changes and context, those
translation models or transformers could pick the incorrect locations
for fixing, because they might not learn what fixing changes are
appropriate in some specific contexts~\cite{icse20}. Finally, the
recent DL-based
approaches~\cite{chen2018sequencer,cure-icse21,lutellier2020coconut,icse22}
extract contextual features to be fed into a single DL model to learn
to fix. However, they do not focus on context learning, despite that
{\em determining the correct context} is crucial to extract the right
features to help a model learn fix context-dependent bugs. Those
approaches do not help their models to correctly learn the contexts
given that the code transformations for bug-fixing are known and could
be used to train the models for context learning.


%\underline{First}, the DL-based APR approaches that learn from {\em
%  bug-fixing patterns}~\cite{white2016deep,gupta2017deepfix} have
%focused on similar code and/or code changes with {\em little or no
%  consideration} on whether those code or fixing patterns appear in
%certain surrounding contexts. \underline{Second}, some DL-based APR
%approaches aim to learn {\em only the code changes} for fixes, e.g.,
%from a buggy subtree in an Abstract Syntax Tree (AST) to the correct
%subtree \cite{chakrabortycodit,see2017get}. In this treatment, too
%little or no context might not help a model learn the fixes that
%depend on a larger surrounding code.

%\underline{Third}, other
%approaches~\cite{hata2018learning,tufano2019learning,tufano2018empirical}
%leveraged machine translation to take the entire buggy method and
%translate it to the correct one, while the fix might be only a small
%editing change to a single statement. Those translation-based
%approaches do not distinguish clearly the boundary of the fixing
%change (e.g., to a buggy statement) and the surrounding context.  Due
%to the mixture of fixing changes and context, those translation models
%or transformers could pick the incorrect locations for fixing, because
%they might not learn what fixing changes are appropriate in some
%specific contexts~\cite{icse20}. \underline{Fourth}, other DL-based
%approaches have separate representations for
%contexts~\cite{chen2018sequencer,cure-icse21,lutellier2020coconut}.
%They extract contextual features to be fed into a single DL model to
%learn to fix.

%\underline{Fifth}, dedicating a separate model for context learning
%has recently been shown to improve over the DL-based approaches with a
%single DL model~\cite{icse20}. In DLFix~\cite{icse20}, the first layer
%is a tree-based RNN model that learns the contexts of bug fixes (CCL)
%and its result is used as an additional weighting input for the second
%layer designed to learn the bug-fixing code transformations
%(CTL). However, the cascading from CCL $\rightarrow$ CTL
%creates a {\em confounding effect} from the inaccuracy of the learning
%of the context to that of the fix transformations.



%===============================================
%the DL-based APR approaches that leverage {\em machine translation or
%  transformers}~\cite{chakrabortycodit,hata2018learning,tufano2018empirical,see2017get}
%often take too little surrounding code as context to learn fixing
%changes or do not have a clear boundary of the fixing changes and the
%context. Let us elaborate this point. Some DL-based APR approaches aim
%to learn {\em only the code changes} for a fix, e.g., from one buggy
%subtree in an Abstract Syntax Tree (AST) or a buggy statement to the
%correct subtree or statement~\cite{chakrabortycodit}. In this
%treatment, too little context might not help a model learn the fixes
%that depend on a larger surrounding code. In contrast, other
%approaches~\cite{chen2018sequencer,hata2018learning} take the entire
%buggy method and translate it to the correct one, while the fix might
%be only a small editing change to a single statement. Those
%translation-based approaches {\em do not distinguish clearly the
%  boundary} of the fixing change (e.g., to a buggy statement) and the
%surrounding context (e.g., the preceding or succeeding code). Due to
%the mixture of fixing changes and context, those machine translation
%models or transformers might not learn what fixing changes are
%appropriate in specific contexts~\cite{icse20}.

%===============================================
%\underline{Third}, to address that issue, DLFix~\cite{icse20} makes a
%clear boundary of fixing changes and the surrounding context, and
%dedicates two layers for those two tasks. The first layer is a
%tree-based RNN model that learns the contexts of bug fixes and its
%result is used as an additional weighting input for the second layer
%designed to learn the bug-fixing code transformations. However, the
%cascading architecture in DLFix from the two layers create a
%confounding effect from the inaccuracy of the learning of the context
%to the learning of the bug-fixing code transformations.
%---------------------------------

%We conjecture that the two tasks of learning the code
%context and learning the bug-fixing code transformations are related
%and dependent on each other. {\bf Correct learning of contexts can
%  benefit the learning of code transformations and vice versa in
%  automated program repair}. For example, in C code, if the
%preceding code (i.e., part of the context) contains \code{fopen} as in
%\code{FILE *fd = fopen (fname, ``rb'');}, then the succeeding
%bug-fixing code is more likely to be \code{if (fd != null)} or
%\code{if (fd == null)}. However, if the preceding code contains
%\code{fread} as in \code{n = fread(buffer,1,size,fd);}, then the
%succeeding bug-fix is more likely to be \code{if (n != size)} or
%\code{if (n == size)}, rather than a null check \code{if (n !=
%  null)}. In contrast, if the bug fix is a
%change from \code{if (fd == null)} into \code{if (fd != null)}, then the
%preceding code more likely contains \code{fopen} than \code{fread}.
%Let us call such relation between two tasks as {\em duality}.

%{\bf Correct learning of contexts can benefit the learning of code
%  transformations and vice versa in automated program repair}

%Tien removed this para
%{\em Moreover, the equally important impact from CTL to CCL is not
%  considered. Such impact from the direction of CTL $\rightarrow$
%CCL, could help the model correctly learn the context, leading to
%better code-transformation learning for context-dependent bugs. In the
%previous example, if the bug fix is a change from \code{if (fd ==
%  null)} into \code{if (fd != null)}, the preceding context more
%likely contains \code{fopen(...)} than \code{fread(...)}.}

%Tien
In this work, we conjecture that to improve the learning of code
transformation for bug-fixing, we {\em dedicate a separate model for
implicit context learning}, rather than explicit feature engineering
for contexts as in the above approaches.
%
We introduce {\tool}, a context-aware, dual-task learning APR
approach, that improves APR via the {\em simultaneous tasks of context
  learning and code transformation learning}.  Context learning and
code-transformation learning are complementary to one another, leading
to better APR. For example, if the preceding context contains
\code{fopen} as in \code{FILE *fd = fopen (fname, ``rb'');}, the
succeeding bug-fixing code is likely \code{if (fd} \code{!=}
\code{null)} or \code{if (fd} \code{==} \code{null)}. However, if the
preceding context contains \code{fread} as in \code{n} \code{=}
\code{fread(buffer,1,size,fd);}, the succeeding fix is likely \code{if
  (n} \code{!=} \code{size)} or \code{if (n} \code{==} \code{size)},
rather than a null check \code{if (n} \code{!=} \code{null)}. In
contrast, if the fix is a change from \code{if (fd} \code{==}
\code{null)} into \code{if (fd} \code{!=} \code{null)}, the key
contextual features are more likely \code{fopen(...)} than
\code{fread(...)}.

\indent In {\tool}, we dedidate two models: CCL for context learning and CTL
for code-transformation learning. We use a model called~{\em
  cross-stitch unit}~\cite{misra2016cross} that connects CTL and
CCL. The sharing of representations between CCL and CTL is modeled
by~learning a linear combination of the input features from two
models. Cross-stitch unit helps regularize CCL and CTL by learning and
enforcing shared representations via combining feature maps. The
rationale for dual-task learning is to propagate the impact of CCL and
CTL and vice versa. {\em The impact from both directions helps both
  models learning better in its own task (better CCL leads to better
  CTL, which leads to better CCL, and so on), and finally leading to
  better generated patches}, which are derived from the CTL's output.

%Tien removed this
%Our idea is that to improve APR for context-dependent bugs, a model
%first needs to have better code-transformation learning for
%bug-fixing. It also needs better context learning, i.e., learning
%contextual features for a fix because incorrect context learning could
%make incorrect fixing for context-dependent bugs.
%%have {\em both better context learning} (learning the correct
%%contextual features for a bug fix) and {\em better code-transformation
%%  learning} (learning the correct modifications).
%For example, in C~code, if the preceding context contains \code{fopen}
%as in \code{FILE *fd = fopen (fname, ``rb'');}, the succeeding
%bug-fixing code is likely \code{if (fd} \code{!=} \code{null)} or
%\code{if (fd} \code{==} \code{null)}. However, if the preceding
%context contains \code{fread} as in \code{n} \code{=}
%\code{fread(buffer,1,size,fd);}, the succeeding fix is more likely
%\code{if (n} \code{!=} \code{size)} or \code{if (n} \code{==}
%\code{size)}, rather than a null check \code{if (n} \code{!=}
%\code{null)}. In contrast, if the fix is a change from \code{if (fd}
%\code{==} \code{null)} into \code{if (fd} \code{!=} \code{null)}, the
%key contextual features are more likely \code{fopen(...)} than
%\code{fread(...)}.

%Tien
%Talk about CTL is the main task

%{\em explicitly models the mutual impact of context learning and code
%  transformation learning in both directions}.
%We train the two models simultaneously with soft-sharing parameters
%to exploit the duality of CCL and CTL.
%
%Specifically, we use a model called {\em cross-stitch
%  unit}~\cite{misra2016cross} that connects CTL and CCL. The sharing
%of representations between CCL and CTL is modeled by learning a linear
%combination of the input features from two models.  Cross-stitch unit
%helps regularize both CCL and CTL by learning and enforcing shared
%representations by combining feature maps.
%This joint training enables the dual-task learning to propagate the
%impact of CCL and CTL and vice~versa.

%Tien removed this
%With dual-task learning, {\tool} has two key departure points to
%address the limitations of the existing DL-based~APR approaches: {\em
%  1) Conceptually, it models both directions of the mutual impacts of
%  CCL and CTL, leading to improve APR; 2) Architecturally, it
%  overcomes the confounding effect in the existing cascading
%  architecture} (Section~\ref{ccl:sec}.2.1). The internal structure of CCL
%and CTL, and their connections in {\tool} are also more advanced than
%DLFix (Section~\ref{eval-methodology:sec}).

For training, the input of the CCL model is the Abstract Syntax Tree
(AST) of a buggy method (as the context) and that of the fixed one.
The input of the CTL model is the AST subtree of each of its buggy
statements, and that of the fixed statement. For
auto-fixing, a fault localization tool is used to identify the buggy
statement(s). The AST subtree of the buggy statement is fed into the
trained~CTL model~to produce the ranked candidate patches, which are
validated via test cases. We leverage a tree-oriented beam search for
efficiency.

%In {\tool}, for training, the input for the CCL model is the Abstract
%Syntax Tree (AST) of a buggy method and that of the fixed method, and
%the input of the CTL model is the AST subtree of each of its buggy
%statements, and that of the respective fixed statement. For
%auto-fixing, a buggy statement to be fixed is identified by a fault
%localization tool. Then, the AST subtree for the buggy statement is
%fed into the trained CTL model to produce the candidate patches. We
%design a novel tree-oriented beam search for efficiency. Finally, the
%candidate patches are ranked and validated via test cases.

We conducted experiments to compare {\tool} with the DL-based APR
models on Defects4J~\cite{defects4j} (395~bugs),
Bugs.jar~\cite{saha2018bugs} (1,158 bugs), and
BigFix~\cite{yioopsla19} (+4.9M methods and 1.8M buggy ones).
%
In Defects4J, {\tool} fixes 194.7\%, 40\%, 27.3\%, 16.7\%, and 3.7\%
more bugs than the baselines: SequenceR~\cite{chen2018sequencer},
DLFix~\cite{icse20}, CoCoNuT~\cite{lutellier2020coconut},
CURE~\cite{cure-icse21}, and DEAR~\cite{icse22}, respectively.
{\tool} complements to the best baseline Recoder in Defects4J but
improves over it in Bugs.jar and BigFix.
%
{\tool} fixed 8 more bugs than the next-best DL model, CURE~\cite{cure-icse21}.
%
Even with the cut-off setting in CURE (taking 10\% more time than
{\tool}), CURE fixed 4 less bugs than {\tool}.
%
In Bugs.jar and BigFix, {\tool} fixes 12.1\% and 14.6\% more bugs than
CURE~\cite{cure-icse21}, using only Top-1 patches,
respectively. Moreover, {\tool} fixed 89 and 45 unique bugs that CURE
missed, while {\tool} missed only 50 and 27 bugs that~were fixed by
CURE in Bugs.jar and BigFix.

The contributions of this paper include:

%are listed as follows:

%{\bf A. DL for APR:} {\tool} is the first DL APR that generates
%comparable and complementary results with powerful pattern-based
%tools, as recently published DL-based APR can only fix very few bugs
%on Defects4J. {\tool} helps confirm that further research on building
%advanced DL to improve APR is promising and valuable.

%{\bf A. A Novel Dual-Task Learning DL-based APR Model:}

%1) Leveraging dual-task learning to propagate the {\bf mutual impacts}
%  in both directions of context and transformation learning.

1) {\tool}: uses dual-task learning to model mutual impacts of~context
learning and transformation learning, leading to better APR.

%2) Adapting/Modifying the cross-stitch unit for dual-task
%learning to work with AST representations in source code.

%2) A demonstration that mutual impacts of context learning
%and code transformation learning lead to better APR.

2) We show that dedicating a model for context learning improves APR
better than extracting contextual features.

%3) A novel {\bf tree-oriented beam search} for candidate generation.

%4) Dual-task architecture avoiding confounding inaccuracies.


%A context-aware APR approach that leverages dual learning {\em to
%  propagate the impact between context learning and bug-fixing code
%  transformation learning to improve APR}. We {\bf advance DL-based
%  APR} with {\bf explicit modeling of the impact of contexts} via {\bf
%  dual learning}.

%{\bf B. Dual Learning Technique for Context Learning.} to explicitly
%model the context and

%with the use of a dual learning scheme that exploits the duality of
%learning bug-fixing code transformations and learning code contexts to
%improve APR performance.

3) Empirical Results: 
%{\bf Advancing DL-based APR approaches with dual-task learning for CCL
%  and CTL}:
%Extensive experiments were performed to evaluate different aspects
%of {\tool} and comparison.
%We evaluated {\tool} against the DL-based models to show that it can
%fix more bugs than those approaches.
Experiments show {\tool}'s improvements over prior
DL-based APRs. Code and data are at~\cite{CDFix2022}.

%to evaluate {\tool} and show it improves
%APR performance over the existing DL-based approaches.

%{\bf B. Empirical Results:} (Code and data are published~\cite{CDFix2022}).
%{\bf Improving over all the DL-based APR approaches}: we evaluated
%{\tool} against the most recent DL-based models to show that it can
%auto-fix more bugs than those state-of-the-art approaches.

%{\tool} is able to detect 2.5 times more bugs than the best performing
%baseline.  {\tool} can fix 253 new bugs (out of 1158 in Bugs.jar) than
%all the other DL-based APR techniques combined.

%-------------------------------------------------------------

%{\bf 1. {\tool}: Novel DL-based fault localization approach} that
%derives the co-change fixing locations for a bug. Our idea is
%to treat such problem as a dual learning task with the joint training
%of the method-level and statement-level co-fixing learning models.

%{\bf 2. Novel graph-based representation learning with co-change
%  statements.} Our graph-based representation learning with GCN
%and the novel type of features in co-change statements enables
%the dual-task models learn derive co-change fixing locations.

%{\bf 3. Extensive empirical evaluation.} We evaluated {\tool} against
%the most recent FL models to show our model's better performance. Our
%replication package is available at~\cite{FixLocator2022}.
