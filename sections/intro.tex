\section{Introduction}

%Fixing software defects is one of the crucial maintenance
%activities. Thus,

Detecting and fixing software defects is one of the most crucial
activities in software development. Thus, several researchers have
developed the approaches to automate the process of bug detecting and
fixing. The approaches that help developers automatically identifying
and fixing a software defect in a program is referred to as {\em
  automated program repair} (APR). The APR approaches can be broadly
classified in the following categories according to their techniques:
%Researchers have proposed several approaches to help developers in
%automatically identifying and fixing the defects in programs. Such
%approaches are referred to as {\em automated program~repair}
%(APR). The APR approaches have been leveraging various techniques in
%the areas of
{\em search-based software engineering}, {\em software mining}, {\em
  machine learning (ML)}, and {\em deep learning (DL)}.

In {\em search-based
  approaches}~\cite{LeGoues-icse12,le2011genprog,martinez2016astor,qi2014strength},
the buggy code is first mutated via certain operators to produce the
potential solution space. Then, a search strategy is designed to find
the fix in the solution space. Instead searching for a solution, the
software mining-based APR models aim to learn the fixing patterns from
the prior bug
fixes~\cite{kim2013automatic,le2016history,liu2019avatar,tbar-issta19,nguyen2013semfix,
  icse10,ray-fse12}. Some mining-based approaches learn the patterns
from the source code~\cite{liu2019avatar,tbar-issta19} and others
learn them from the bug-fixing code
changes~\cite{wen2018context,Simfix,koyuncu2018fixminer}.  The fixing
patterns/templates could be mined automatically from the repositories
or pre-defined via semi-automated
techniques~\cite{le2016history,nguyen2013semfix,liu2019avatar,tbar-issta19}.

Recent advances in machine learning have enabled several models to
implicitly learn from the bug fixes to apply to repair the current
buggy code.  They then derive the candidate fixes and rank them
according to their
likelihoods~\cite{long2016automatic,long2017automatic,saha2017elixir}.
With the recent revolutionary advances in deep learning (DL), software
engineering researchers have leverage them to automatically derive the
fixing changes to a given buggy code. Some DL-based approaches learn
the fix from prior similar bug fixes and/or
patterns~\cite{gupta2017deepfix,white2019sorting,white2016deep}.
Other DL-based approaches treat APR as machine translation from a
buggy version to a correct one with some DL models including
transformers and
others~\cite{chakrabortycodit,chen2018sequencer,hata2018learning,tufano2018empirical,see2017get,tufano2019learning}. Instead
of treating APR as machine translation, other DL-based approaches aim
to implicitly learn the transformation rules from a buggy code to a
correct one accordingly to the surrounding context of the
transformations~\cite{icse20}.


%While some DL-based APR approaches learn similar
%fixes~\cite{gupta2017deepfix,white2019sorting,white2016deep}, other
%ones
%use machine translation or neural network models with
%various~code~abstractions to generate
%patches~\cite{chakrabortycodit,chen2018sequencer,hata2018learning,tufano2018emp%irical,see2017get,tufano2019learning,icse20}.


%The fixing patterns/templates could be mined automatically from the
%repositories or pre-defined via semi-automated
%techniques~\cite{le2016history,nguyen2013semfix,liu2019avatar,tbar-issta19}.
%

%Other approaches use software mining to {\em mine and learn fixing
% patterns} from prior bug fixes
%\cite{kim2013automatic,le2016history,liu2019avatar,tbar-issta19,nguyen2013semfix} or cloned code~\cite{icse10,ray-fse12}. Fixing patterns are at the
%source code level \cite{liu2019avatar,tbar-issta19} or at the change
%level~\cite{wen2018context,Simfix,koyuncu2018fixminer}.  {\em Machine
%  learning} has been used to mine fixing patterns and the candidate
%fixes are ranked according to their
%likelihoods~\cite{long2016automatic,long2017automatic,saha2017elixir}.
  
