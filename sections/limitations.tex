\subsection{Limitations}
\label{sec:limitations}

<<<<<<< HEAD
{\tool} still has the following limitations. 
%First, it fixes only single-statement bugs. Second, 
First, it could not produce a fix with an
arbitrarily large size because it sets a limit with $P$ and $Q$ in the
size of the fixed subtree. Third, it does work well for the cases of
only inserting or deleting code. Fourth, {\tool} learns only the
tree-based code transformation, thus, has limitations in learning
program semantic changes involving complex program dependencies, etc.
=======
{\tool} still has the following limitations. First, it does not
determine the faulty statement to be fixed. As with existing work
CURE~\cite{cure-icse21} and CoCoNuT~\cite{lutellier2020coconut}, a
common usage of {\tool} is that a user uses a fault localization tool
to detect a faulty statement or (s)he can pinpoint the faulty one and
invoke {\tool} for an auto-fixing suggestion. Second, {\tool} cannot
produce a fix with an arbitrarily large size because its DL model does
not learn well large tree structure changes. Third, {\tool} cannot fix
a bug with only insertions of new statements, and it does not work
well with a fix with only deletions.
%because it sets a limit with $P$ and $Q$ in the size of the fixed
%subtree.
Finally, {\tool} learns only the tree-based code transformations for
APR. The changes in program semantics such as changes to PDGs, CFGs,
etc. could help improve APR performance in our model.
>>>>>>> 2e9f521a478ec7b7664299b3ea50fa590ce11c47
