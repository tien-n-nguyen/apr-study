\subsection{Limitations}
\label{sec:limitations}


%{\tool} still has the following limitations.

%First, {\tool} does not determine the faulty statements to be fixed.
%A common usage of {\tool} is that one uses a fault localization tool
%to detect the faulty statements or (s)he can pinpoint the faulty one
%and invoke {\tool} for auto-fixing. Second,
{\tool} cannot produce a fix with an arbitrarily large size because
its CTL model does not learn well large tree-structured
changes. Second, it cannot fix a bug with only insertions of new
statements, and it does not work well with a fix with only deletions.
%because it sets a limit with $P$ and $Q$ in the size of the fixed
%subtree.
%It cannot fix multi-statement bugs where the changes depend on one
%another.
%
Third, {\tool} learns only the tree-based code transformations for
APR. The changes in program semantics such as changes to PDGs, CFGs,
etc. could help improve APR performance in our model. Fourth, the
produced fix could contain syntactically incorrect code. We could
integrate the generation of template code as in Recoder and then
select the variables in the valid scope in the program. Finally,
{\tool} does not consider the test execution information as in
RewardRepair. This could be a nice integration in future work.

