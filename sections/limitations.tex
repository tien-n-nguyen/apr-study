\subsection{Limitations}
\label{sec:limitations}


%{\tool} still has the following limitations.
First, {\tool} does not determine the faulty statements to be fixed.
%As with some existing work~\cite{cure-icse21,lutellier2020coconut},
A common usage of {\tool} is that one uses a fault localization tool
to detect the faulty statements or (s)he can pinpoint the faulty one
and invoke {\tool} for auto-fixing. Second, it cannot produce a fix
with an arbitrarily large size because its CTL model does not learn
well large tree-structured changes. Third, it cannot fix a bug with
only insertions of new statements, and it does not work well with a
fix with only deletions.
%because it sets a limit with $P$ and $Q$ in the size of the fixed
%subtree.
%It cannot fix multi-statement bugs where the changes depend on one
%another.
%
Finally, {\tool} learns only the tree-based code transformations for
APR. The changes in program semantics such as changes to PDGs, CFGs,
etc. could help improve APR performance in our model.

